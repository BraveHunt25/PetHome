% !TeX root = ../proyecto.tex
%--------------------------------------
\section{IU5 Pantalla Home Cliente}

\subsection{Objetivo}
	\cdtInstrucciones El objetivo de esta pantalla es permitir al cliente ver sus reservacionses, para que de esta forma tenga la opcion de editarlas o eliminarlas, y tambien que tenga la opcion de hacer una reservacion .

\subsection{Diseño}
	\cdtInstrucciones Esta pantalla {UI5}{Pantalla Home Cliente} aparece al momento de registrarse o iniciar sesion, una vez dentro tenemos la opcion de hacer una reservacion dando click en el boton \IUbutton{Reservar}, asi como visualizar nuestras reservaciones y tener la opcion de editar o eliminar la reservacion con los botones \IUbutton{Editar},\IUbutton{Eliminar}.

\IUfig[.7]{Reservaciones}{IU5}{Pantalla de Home Cliente}
\IUfig[.7]{sinReservaciones}{IU5}{Pantalla de Home Cliente sin reservaciones}

%\IUfig[ancho de la figura: valor entre 1 y .1]{Nombre corto de la pantalla sin espacions ni acentos}{IUXX}{Nombre largo de la pantalla.}

\subsection{Salidas}

	%\cdtInstrucciones{Liste las salidas de la interfaz. Si coinciden con las del caso de uso solo indiquelo. Esta ,ista debe incluir los mensajes}

	\begin{itemize}
		\item Si no se ha echo ninguna reservacion, se mostrara el mensaje \MSGref{MSG-003}{No hay reservaciones todavia}.
	\end{itemize}
	
\subsection{Entradas}

	%\cdtInstrucciones{Liste las entradas de la interfaz. Si coinciden con las del caso de uso solo indiquelo.}
	\begin{itemize}
		\item Ninguna
	\end{itemize}

\subsection{Comandos}
	%\cdtInstrucciones{Describa cada control (botónes, areas de drag and drop, componentes interactivos, animaciones, etc.) que se puede utilizar dentro de la pantalla indicando o que hacen y si cambia de pantalla.}

\begin{itemize}
	\item \IUbutton{Hacer reservacion}: Muestra la pantalla \IUref{UI2}{Pantalla de Datos Obligatorios de la Mascota}.
	\item \IUbutton{Editar}: Muestra la pantalla \IUref{UI2}{Pantalla de Datos Obligatorios de la Mascota}, con los datos de esa reservacion y permite modificarlos.
	\item \IUbutton{Eliminar}: Elimina la reservación.
\end{itemize}

