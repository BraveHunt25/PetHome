% !TeX root = ../proyecto.tex
%--------------------------------------
\section{IU2 Pantalla de Datos Obligatorios de la Mascota }

\subsection{Objetivo}
	\cdtInstrucciones El objetivo de esta pantalla es permitir al usuario capturar los datos que son obligatorios de su mascota.

\subsection{Diseño}
	\cdtInstrucciones Esta pantalla \IUref{IU3}{Registrar Mascota} aparece al momento de querer hacer una reservacion, para ingresar los datos de la mascota basta con llenar la informacion desde teclado o mouse, una vez llenados todos los datos podemos hacer click en el boton \IUbutton{Registrar}   

\IUfig[.7]{RegistrarDatosMascota}{IU3}{Registrar Datos Mascota}

%\IUfig[ancho de la figura: valor entre 1 y .1]{Nombre corto de la pantalla sin espacions ni acentos}{IUXX}{Nombre largo de la pantalla.}

\subsection{Salidas}

	\begin{itemize}
		\item Si los datos de esta pantalla no son llenados, al momento de dar click en el boton \IUbutton{Siguiente}, saltara el mensaje \MSGref{MSG-001}{Campo obligatorio}.
	\end{itemize}
	
\subsection{Entradas}
	\begin{itemize}
		\item Las entradas son las del caso de uso \hyperlink{CU3}{CU3 Registrar Mascota}.
		\item Nombre de la Mascota
		\item Tamaño
		\item Fecha de nacimiento
		\item Especie
		\item Raza
		\item Color de la Mascota
		\item RUAC
		\item Vacunas
	\end{itemize}

\subsection{Comandos}
	%\cdtInstrucciones{Describa cada control (botónes, areas de drag and drop, componentes interactivos, animaciones, etc.) que se puede utilizar dentro de la pantalla indicando o que hacen y si cambia de pantalla.}

\begin{itemize}
	\item \IUbutton{Siguiente}: Verifica que los datos hayan sido llenados. Si la verificación es correcta, se muestra la \IUref{UI3}{Pantalla de Datos Extra de la Mascota}.
\end{itemize}

