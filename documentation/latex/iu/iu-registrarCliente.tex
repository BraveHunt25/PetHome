% !TeX root = ../proyecto.tex
%--------------------------------------
\section{IU1 Registrar Cliente}

\subsection{Objetivo}
	\cdtInstrucciones  El objetivo de esta pantalla es permitir al cliente capturar sus datos personales, registrarse en el sistema y poder acceder a las operaciones del sistema definidas para su perfil .


\subsection{Diseño}
	\cdtInstrucciones Esta pantalla \IUref{IU1}{Registrar Cliente} aparece cuando el cliente solicite registarse, para ingresar sus datos personales basta con llenar la informacion desde teclado o mouse, una vez llenados todos los datos podemos hacer click en el boton \IUbutton{Registrar} . 

\IUfig[.7]{RegistroCliente}{IU1}{Registrar Cliente.}

%\IUfig[ancho de la figura: valor entre 1 y .1]{Nombre corto de la pantalla sin espacions ni acentos}{IUXX}{Nombre largo de la pantalla.}

\subsection{Salidas}

	

	\begin{itemize}
		\item Las salidas son las mismas que las del caso de uso \hyperlink{CU1}{CU1 Registrar Cliente}.
	\end{itemize}
	
\subsection{Entradas}

	\cdtInstrucciones{Liste las entradas de la interfaz. Si coinciden con las del caso de uso solo indiquelo.}
	\begin{itemize}
		\item Las entradas son las mismas que las del caso de uso \hyperlink{CU1}{CU1 Registrar Cliente}.
	\end{itemize}

\subsection{Comandos}
	\cdtInstrucciones{Describa cada control (botónes, areas de drag and drop, componentes interactivos, animaciones, etc.) que se puede utilizar dentro de la pantalla indicando o que hacen y si cambia de pantalla.}

\begin{itemize}
	\item \IUbutton{Registrar}: Verifica que los datos personales hayan sido llenados correctamente. Si la verificación es correcta, se muestra la \IUref{UI5}{Pantalla de Home Cliente sin reservaciones}.
	\item \IUbutton{Iniciar sesion}: Muestra la patalla \IUref{IU1.2}{Iniciar sesion}.
\end{itemize}

