% !TeX root = proyecto.tex


\cdtInstrucciones{En esta sección describa todas las reglas de negocio identificadas.}


% Tipo: \btDerivation (no aplica Clase), \btEnabler, \btTimer, \btExecutive
% Clase: \bcCondition, \bcIntegrity, \bcAutorization.
% Cumplimiento: \blStrict \blDeferred \blPreAutorized \blPostJustified \blOverride \blGuideline
\begin{BussinesRule}[%
	\brClassification{\btEnabler}{\bcCondition}{\blStrict}
	]{BR-001}{Nombre de la regla de negocio}
	
				% Opciones para nivel: \blControlling, \blInfluencing
	\BRitem[Descripción:] Descripción de la regla. Forma coloquial a manera de reglamento.
	\BRitem[Motivación:] Describa por que es importante la regla.
	\BRitem[Sentencia:] Sentencia formal de la regla.
	\BRitem[Ejemplo positivo:] Indique uno o varios ejemplos en donde la regla se cumple.
        \begin{itemize}
        	\item ...
        \end{itemize}
	
	\BRitem[Ejemplo negativo:] Indique uno o varios ejemplos en dónde la regla no se cumple.
		\begin{itemize}
        	\item ...
        \end{itemize}
	
	\BRitem[Referenciado por:] Liste los casos de uso en donde la regla no se cumple. por ejemplo \hyperlink{CUCE3.2}{CUCE3.2}, \hyperlink{CUCE3.3}{CUCE3.3}.
\end{BussinesRule}


\begin{BussinesRule}[%Cambiar el nivel de la regla a \blPostJustified
	\brClassification{\btEnabler}{\bcCondition}{\blStrict}
	]{BR-002}{Vacunas Obligatorias para la Mascota}
	\BRitem[Descripción:] Para que una mascota pueda hacer uso de las instalaciones del hotel, debe contar con las 3 vacunas obligatorias: Rabia, Bordetella, Leptospirosis
%En la PROFECO menciona que para perros deben ser: parvovirus canino, moquillo, hepatitis canina y la rabia, para gatos: Rabia, Trivalente para rinotraqueitis, calicivirus y panleucopenia y leucemia felina. La vacunación para hurones y conejos es cubierta con la de gatos. Para cerdos: parvovirus, erysipelothrix rhusiopathiae y aujeszky.
	\BRitem[Motivación:] Al darse la convivencia entre especies y distintos estilos de vida, es mandatorio legalmente la aplicaciones de determinadas vacunas en función de la especie para garantizar seguridad sanitaria a todos los huéspedes y sus dueños.
	\BRitem[Sentencia:] Sea $a$ una mascota, $Especies$ el conjunto de especies que puede hospedar el hotel, $Hospedados$ el conjunto de mascotas que hospeda el hotel, $Vacunas$ el conjunto de vacunas aplicables a culquier mascota, $Obligatorias$ el subconjunto de $Vacunas$ ($Obligatorias\subset Vacunas$) de vacunas obligatorias, $ultimaAplicacion$ la fecha de la última aplicación de la vacuna, $fechaHospedaje$ la fecha en la que iniciará su hospedaje y $periodoVigente$ el tiempo de vigencia de la vacuna; $a \in Hospedados \iff a\in (Especies\cap  Obligatorias) \land (fechaHospedaje - ultimaAplicacion < periodoVigencia)$, es decir, una mascota estará hospedada si y solo si pertenece al conjunto de especies con vacunas obligatorias aplicadas y la diferencia de la fecha del check-in y la de aplicación de la vacuna sea menor al periodo de vigencia de la vacuna misma.
	\BRitem[Ejemplo positivo:] El perro Rufus es hospedado ya que cuenta con las vacunas :
	\begin{itemize}
        		\item Rabia: aplicada hace 2 meses.
		\item Bordetella: aplicada hace 3 meses.
		\item Leptospirosis: hace 4 meses .
	\end{itemize}
	
	\BRitem[Ejemplo negativo:] El perro Alemán no es hospedado ya que solo cuenta con las vacunas:
	\begin{itemize}
        		\item Rabia: aplicada hace 12 meses.
		\item Bordetella: aplicada hace 14 meses.
	\end{itemize}
	necesitando una consulta con el médico veterinario
	
	\BRitem[Referenciado por:] \hyperlink{CU3}{CU3}.
\end{BussinesRule}