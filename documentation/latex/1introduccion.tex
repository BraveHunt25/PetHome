% !TeX root = proyecto.tex

%=========================================================
\chapter{Introducción}


\cdtInstrucciones{
	Presentar el documento, indicando su contenido, a quien va dirigido, quien lo realizó, por que razón, dónde y cuando. \\
}
	Este documento contiene el análisis de requerimientos del proyecto ``{\em Nombre del proyecto}'' que servirá como base para el análisis, diseño, construcción, pruebas y aceptación del proyecto.

%---------------------------------------------------------
\section{Presentación}


\cdtInstrucciones{
	Presente en un par de párrafos el contexto y el problema en que se define el proytecto.
}

\cdtInstrucciones{
	Indique el propósito del documento, a quien va dirigido y como debe ser utilizado el documento.
}
	
%---------------------------------------------------------
\section{Organización del contenido}

\cdtInstrucciones{
	Indique el contenido y organización del documento.
}
	En el capítulo \ref{cap:reqUsr} ...
	
	En el capítulo \ref{cap:reqSist} ...

%---------------------------------------------------------
\section{Notación, símbolos y convenciones utilizadas}

\cdtInstrucciones{
	Indique la notacion utilizada así como nórmas o estándares de documentación utilizados en el documento.
}	

