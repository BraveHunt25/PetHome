%!TEX root = ../proyecto.tex

% Plantilla para caso de uso sencillo con ejemplos de comandos e intrucciones.
%-------------------------------------- COMIENZA descripción del caso de uso.

%\begin{UseCase}[archivo de imágen]{UCX}{Nombre del Caso de uso}{
%--------------------------------------
\begin{UseCase}{CU2}{Registrar Reservación}{
		Cuando un cliente quiere realizar una reservación en hotel, debe registrar sus datos de reservación, una vez registrados se creara la reservación y se mostraran los detalles de la misma.
	}
	\UCitem{Versión}{\color{Gray}
		0.3	% Ponga un número de versión, 
	}\UCitem{Autor}{\color{Gray}
		Hernández Jiménez Erick Yael. % Analista responsable de especificar el CU
	}\UCitem{Supervisa}{\color{Gray}
		Ruiz Evaristo Marco Antonio. % Analista responsable de verificar que está correcto.
		% TODO: Dar de alta al actor Usuario
	}\UCitem{Actor}{
		\hyperlink{Cliente}{Cliente}% No olvide dar de alta el actor.
	}\UCitem{Propósito}{\begin{Titemize}%Indique los fines, objetivos, propósitos o valores agregados del Caso de uso.
		\Titem Proporcionar a los clientes un medio conveniente para realizar reservaciones para sus mascotas, permitiendo planificar sus necesidades de cuidado con anticipación.
		\Titem Gestionar la disponibilidad de las reservaciones, evitando conflictos y asegurando que los recursos estén bien distribuidos.
		\Titem Asegurar que todos los datos relevantes se ingresen de manera correcta y completa para el uso efectivo de los servicios.
		\Titem Ofrecer un medio para enviar recordatorios al cliente sobre las reservas próximas, reduciendo así el riesgo de cancelaciones tardías o ausencias.
		\end{Titemize}
	}\UCitem{Entradas}{\begin{Titemize}
			\Titem \hyperlink{SUCURSAL.NOMBRE_SUCURSAL}{Sucursal}.
			\Titem \hyperlink{DISPONIBILIDAD.CHECK-IN}{Check-in}
			\Titem \hyperlink{DISPONIBILIDAD.CHECK-OUT}{Check-out}.
			\Titem \hyperlink{ESPECIE.NOMBRE_ESPECIE}{Especies} de las mascotas a hospedar.
			\Titem \hyperlink{RAZA.NOMBRE_RAZA}{Razas} de las mascotas a hospedar.
			\Titem Número de mascotas de la misma raza que el cliente desea hospedar.
			\Titem \hyperlink{TAMANO.TAMANO}{Tamaños} de las mascotas a hospedar.
			\Titem \hyperlink{TIPO_CUARTO.NOMBRE_CUARTO}{Nombre de los cuartos} que desea reservar el cliente.
			\Titem \hyperlink{TIPO_SERVICIO.NOMBRE_SERVICIO}{Nombre de los servicios} que desea adquirir para la mascota el cliente.
		\end{Titemize}
	}\UCitem{Origen}{\begin{Titemize}
			\Titem \hyperlink{SUCURSAL.NOMBRE_SUCURSAL}{Sucursal}: Menú desplegable en la pantalla \IUref{IU6}{Búsqueda de hospedajes}.
			\Titem \hyperlink{DISPONIBILIDAD.CHECK-IN}{Check-in}: Selector de fechas en la pantalla \IUref{IU6}{Búsqueda de hospedajes}.
			\Titem \hyperlink{DISPONIBILIDAD.CHECK-OUT}{check-out}: Selector de fechas en la pantalla \IUref{IU6}{Búsqueda de hospedajes}.
			\Titem \hyperlink{ESPECIE.NOMBRE_ESPECIE}{Especies}: Menú desplegable en la pantalla \IUref{IU6}{Búsqueda de hospedajes}.
			\Titem \hyperlink{RAZA.NOMBRE_RAZA}{Razas}: Menú desplegable en la pantalla \IUref{IU6}{Búsqueda de hospedajes}.
			\Titem \hyperlink{TAMANO.TAMANO}{Tamaños}: Menú desplegable en la pantalla \IUref{IU6}{Búsqueda de hospedajes}.
			\Titem \hyperlink{TIPO_CUARTO.NOMBRE_CUARTO}{Nombre de los cuartos}: Botones de selección en la pantalla \IUref{IU7}{Catálogo de habitaciones}.
			\Titem \hyperlink{TIPO_SERVICIO.NOMBRE_SERVICIO}{Nombre de servicios}: Menú desplegable en la pantalla \IUref{IU9}{Servicios adicionales}.
		\end{Titemize}
	}\UCitem{Salidas}{\begin{Titemize}
			\Titem \hyperlink{SUCURSAL.NOMBRE_SUCURSAL}{Sucursal}.
			\Titem \hyperlink{DISPONIBILIDAD.CHECK-IN}{Check-in}.
			\Titem \hyperlink{DISPONIBILIDAD.CHECK-OUT}{Check-out}.
			\Titem \hyperlink{ESPECIE.NOMBRE_ESPECIE}{Especies} de las mascotas a hospedar.
			\Titem \hyperlink{RAZA.NOMBRE_RAZA}{Razas} de las mascotas a hospedar.
			\Titem \hyperlink{TAMANO.TAMANO}{Tamaños} de las mascotas a hospedar.
			\Titem \hyperlink{TIPO_CUARTO.NOMBRE_CUARTO}{Nombre de los cuartos} que desea reservar el cliente.
			\Titem \hyperlink{TIPO_CUARTO.PRECIO_CUARTO}{Precio del los cuartos} que desea reservar el cliente.
			\Titem \hyperlink{SUCURSAL.CALLE_DIRECCION}{Calle} de la sucursal en la que reserva el cliente.
			\Titem \hyperlink{SUCURSAL.NUMERO_DIRECCION}{Número de calle} de la sucursal en la que reserva el cliente.
			\Titem \hyperlink{SUCURSAL.NOMBRE_COLONIA_DIRECCION}{Colonia} de la sucursal en la que reserva el cliente.
			\Titem \hyperlink{SUCURSAL.NOMBRE_MUNICIPIO_DIRECCION}{Municipio} de la sucursal en la que reserva el cliente.
			\Titem \hyperlink{SUCURSAL.CODIGO_POSTAL_DIRECCION}{Código postal} de la sucursal en la que reserva el cliente.
			\Titem \hyperlink{SUCURSAL.ENTIDAD_DIRECCION}{Entidad} de la sucursal en la que reserva el cliente.
			\Titem \hyperlink{TIPO_SERVICIO.NOMBRE_SERVICIO}{Nombre de los servicios} que desea adquirir para la mascota el cliente.
			\Titem \hyperlink{TIPO_SERVICIO.COSTO_BASE}{Costo base de los servicios} que desea adquirir para la mascota el cliente.
			\Titem \hyperlink{TIPO_SERVICIO.COSTO_EXTRA}{Costo extra de los servicios} que desea adquirir para la mascota el cliente en caso de presentarse la necesidad de servicios emergentes.
			\Titem \hyperlink{CORTE_RESERVACION.SUMA_PRECIO_CUARTOS}{Total de los cuartos} que contrata el cliente.
			\Titem \hyperlink{CORTE_RESERVACION.SUMA_PRECIO_CITAS_SERVICIOS}{Total de los servicios} que contrata el cliente.
			\Titem \hyperlink{CORTE_RESERVACION.SUMA_TOTAL_RESERVACION}{Suma total por el hospedaje} que contrata el cliente.
		\end{Titemize}
	}\UCitem{Destino}{\begin{Titemize}
			\Titem \hyperlink{SUCURSAL.NOMBRE_SUCURSAL}{Sucursal}: En la pantalla  \IUref{IU7}{Catálogo de habitaciones} y \IUref{IU8}{Vista general del cuarto}.
			\Titem \hyperlink{DISPONIBILIDAD.CHECK-IN}{Check-in}: En la pantalla \IUref{IU7}{Catálogo de habitaciones} y \IUref{IU8}{Vista general del cuarto}.
			\Titem \hyperlink{DISPONIBILIDAD.CHECK-OUT}{check-out}: En la pantalla \IUref{IU7}{Catálogo de habitaciones} y \IUref{IU8}{Vista general del cuarto}.
			\Titem \hyperlink{ESPECIE.NOMBRE_ESPECIE}{Especies}: En la pantalla \IUref{IU7}{Catálogo de habitaciones} y \IUref{IU8}{Vista general del cuarto}.
			\Titem \hyperlink{RAZA.NOMBRE_RAZA}{Razas}: En la pantalla \IUref{IU7}{Catálogo de habitaciones} y \IUref{IU8}{Vista general del cuarto}.
			\Titem \hyperlink{TAMANO.TAMANO}{Tamaños}: En la pantalla \IUref{IU7}{Catálogo de habitaciones} y \IUref{IU8}{Vista general del cuarto}.
			\Titem \hyperlink{TIPO_CUARTO.NOMBRE_CUARTO}{Nombre de los cuartos}: En la pantalla \IUref{IU7}{Catálogo de habitaciones} y \IUref{IU8}{Vista general del cuarto}.
			\Titem \hyperlink{TIPO_CUARTO.PRECIO_CUARTO}{Precio de los cuartos}: En la pantalla \IUref{IU7}{Catálogo de habitaciones} y \IUref{IU8}{Vista general del cuarto}.
			\Titem \hyperlink{SUCURSAL.CALLE_DIRECCION}{Calle}: En un mapa en la pantalla \IUref{IU8}{Vista general del cuarto}.
			\Titem \hyperlink{SUCURSAL.NUMERO_DIRECCION}{Número de calle}: En un mapa en la pantalla \IUref{IU8}{Vista general del cuarto}.
			\Titem \hyperlink{SUCURSAL.NOMBRE_COLONIA_DIRECCION}{Colonia}: En un mapa en la pantalla \IUref{IU8}{Vista general del cuarto}.
			\Titem \hyperlink{SUCURSAL.NOMBRE_MUNICIPIO_DIRECCION}{Municipio}: En un mapa en la pantalla \IUref{IU8}{Vista general del cuarto}.
			\Titem \hyperlink{SUCURSAL.CODIGO_POSTAL_DIRECCION}{Código postal}: En un mapa en la pantalla \IUref{IU8}{Vista general del cuarto}.
			\Titem \hyperlink{SUCURSAL.ENTIDAD_DIRECCION}{Entidad}: En un mapa en la pantalla \IUref{IU8}{Vista general del cuarto}.
			\Titem \hyperlink{TIPO_SERVICIO.NOMBRE_SERVICIO}{Nombre de los servicios}: En la pantalla \IUref{IU9}{Servicios adicionales}.
			\Titem \hyperlink{TIPO_SERVICIO.COSTO_BASE}{Costo base de los servicios}: En la pantalla \IUref{IU9}{Servicios adicionales}.
			\Titem \hyperlink{TIPO_SERVICIO.COSTO_EXTRA}{Costo extra de los servicios}: En la pantalla \IUref{IU9}{Servicios adicionales}.
			\Titem \hyperlink{CORTE_RESERVACION.SUMA_PRECIO_CUARTOS}{Total de los cuartos}: En la pantalla \IUref{IU9}{Servicios adicionales} y en la pantalla \IUref{IU11}{Resumen del hospedaje}.
			\Titem \hyperlink{CORTE_RESERVACION.SUMA_PRECIO_CITAS_SERVICIOS}{Total de los servicios}: En la pantalla \IUref{IU9}{Servicios adicionales} y en la pantalla \IUref{IU11}{Resumen del hospedaje}.
			\Titem \hyperlink{CORTE_RESERVACION.SUMA_TOTAL_RESERVACION}{Suma total por el hospedaje}: En la pantalla \IUref{IU9}{Servicios adicionales} y en la pantalla \IUref{IU11}{Resumen del hospedaje}.
		\end{Titemize}
	}\UCitem{Precondiciones}{\begin{Titemize}
			% Incluya Precondiciones lógicas, de negocio e incluso las que debe atender el usuario. 
			\Titem Debe haber al menos un cuarto en al menos una sucursal.
			\Titem El cliente debe acceder al sistema por el caso de uso \UCref{CUX}{Iniciar Sesion} o \UCref{CU1}{Registrar Cliente}.
			\Titem Debe haber disponibilidad de habitaciones en las fechas deseadas.
		\end{Titemize}
	}\UCitem{Postcondiciones}{\begin{Titemize}
			\Titem Habrá una nueva reservación registrada para este cliente en el sistema.
			\Titem El calendario de disponibilidad se ha actualizado para reflejar las fechas reservadas.
			\Titem El cliente ha recibido una confirmación de la reservación a traves de la pantalla \IUref{IU5}{Home Cliente} y de un correo electrónico.
			\Titem Se habilitará el botón \IUbutton{Mascotas} en la pantalla \IUref{IU5}{Home Cliente}
			\Titem Todas las mascotas que el cliente haya asociado a la reservación serán accesibles para el cliente en la pantalla \IUref{IUX}{Consultar mascotas propias}.
		\end{Titemize}
	}\UCitem{Errores}{\begin{Titemize}
			\Titem {\bf \hypertarget{CU2.E1}{E1}}: Cuando el cliente no haya iniciado sesión en el sistema, se le indica que es \MSGref{MSG-013}{Necesario iniciar sesión} y dirige al cliente al caso de uso \IUref{CU1}.
			\Titem {\bf \hypertarget{CU2.E2}{E2}}: Cuando el cliente no haya ingresado los datos mínimos necesarios para la operación, se le muestra el mensaje  \MSGref{MSG-001}{Campo obligatorio} y regresa al paso anterior.
			\Titem {\bf \hypertarget{CU2.E3}{E3}}: Cuando el cliente ingrese un tipo de dato diferente al solicitado, el sistema muestra el mensaje \MSGref{MSG-007}{Dato no válido} y regresa dos pasos antes.
			\Titem {\bf \hypertarget{CU2.E4}{E4}}: Cuando no haya cuartos en el periodo indicado con las características que el cliente solicita, el sistema muestra el mensaje \MSGref{MSG-008}{Fecha no disponible} y dirige al cliente a la pantalla \IUref{IU6}{Búsqueda de hospedajes}.
			\Titem {\bf \hypertarget{CU2.E5}{E5}}: Cuando el cliente solicite la información de un cuarto al que no se puede acceder, el sistema muestra el mensaje \MSGref{MSG-014}{Detalles del cuarto no disponibles} y actualiza la pantalla \IUref{IU7}{Catálogo de habitaciones}.
			\Titem {\bf \hypertarget{CU2.E6}{E6}}: Cuando no haya disponibilidad de servicios extra en el periodo indicado con las características que el cliente solicita, el sistema muestra el mensaje \MSGref{MSG-015}{Servicios no disponibles}.
		\end{Titemize}
	}\UCitem{Tipo}{
		Caso de uso primario
	}\UCitem{Observaciones}{
		Se considera que la disponibilidad de habitaciones y servicios no cambia durante la realización del registro de reservación.
	}
\end{UseCase}

%--------------------------------------
\begin{UCtrayectoria}
	\UCpaso[] El cliente selecciona  el botón \IUbutton{Realizar reservación} de la pantalla \IUref{IU5}{Home Cliente}.
	\UCpaso[] El sistema verifica que el cliente ya se encuentre ingresado en el sistema con un perfil de usuario \ErrorRef{CU2}{E1}{Necesario iniciar sesión}.
	\UCpaso[] El sistema solicita los datos para buscar cuartos que sean de interés para cliente mediante la pantalla \IUref{IU6}{Búsqueda de hospedajes}.
	\UCpaso[] El cliente introduce los datos para buscar cuartos en la sucursal de su preferencia y selecciona el botón \IUbutton{Buscar}.
	\UCpaso[] El sistema verifica que se hayan introducido todos los datos necesarios para la búsqueda \ErrorRef{CU2}{E2}{Campo obligatorio}.
	\UCpaso[] El sistema verifica que se hayan introducido datos válidos para la búsqueda \ErrorRef{CU2}{E3}{Dato no válido}.
	\UCpaso[] El sistema verifica que se haya cuartos disponibles en el periodo y con todas las características solicitadas \ErrorRef{CU2}{E4}{Fecha no disponible}.
	\UCpaso[] \label{muestra_catalogo}El sistema muestra todos los cuartos que cumplan con los requisitos deseados, mostrando el mensaje \MSGref{MSG-016}{Cuartos para mascotas} en la pantalla \IUref{IU7}{Catálogo de habitaciones}.
	\UCpaso[] El cliente selecciona el botón \IUbutton{Seleccionar} correspondiente al cuarto de interés en la pantalla \IUref{IU7}{Catálogo de habitaciones}.
	\UCpaso[] El sistema verifica que se puedan acceder a los datos del cuarto de interés \ErrorRef{CU2}{E5}{Detalles de cuarto no disponibles} y dirige al cliente a la pantalla \IUref{IU8}{Vista general del cuarto}.
	\UCpaso[] El cliente ingresa el número de mascotas de la misma raza y tamaño que desea asociar al mismo cuarto.
	\UCpaso[] El sistema verifica que se hayan introducido todos los datos necesarios para la reservación \ErrorRef{CU2}{E2}{Campo obligatorio}.
	\UCpaso[] El sistema verifica que se hayan introducido datos válidos para la reservación\ErrorRef{CU2}{E3}{Dato no válido}.
	\UCpaso[] Se repite desde el paso \hyperlink{muestra_catalogo}{7} hasta haber reservado para todas las mascotas que haya introducido el cliente inicialmente.
	\UCpaso[] El sistema dirige al cliente a la pantalla \IUref{IU9}{Servicios adicionales} con los campos necesarios para agregar cita a tantas mascotas como haya indicado inicialmente y desee mientras haya servicios disponibles \ErrorRef{CU2}{E6}{Servicios no disponibles}.
	\UCpaso[] El cliente indica al sistema cuántos y qué servicios desea contratar, actualizando los costos totales en la pantalla tras cada modificación.
	\UCpaso[] El cliente indica al sistema que desea completar su reservación con el botón \IUbutton{Reservar}.
	\UCpaso[] \label{CU3_Reg_mascota}El sistema ofrece al cliente la posibilidad de dar más detalles de sus mascotas asociadas a la reservación \IUref{CU3}{Registrar Mascota} mediante la pantalla \IUref{IU10}{Registrar más detalles}.
	\UCpaso[] El sistema dirige al cliente a la pantalla de \IUref{IU11}{Resumen del hospedaje}.
	\UCpaso[] El cliente indica que desea proceder al pago con el botón \IUbutton{Pagar}.
	\UCpaso[] \label{CU2_Reg_pago}El sistema muestra la pantalla \IUref{IU5}{Registrar Pago}.
	\UCpaso[] El sistema registra la reservación y muestra la pantalla \IUref{IU5}{Home Cliente} con los detalles de la reservación.


	
\end{UCtrayectoria}


%--------------------------------------
% Las trayectorias alternativas se identifican con Letras: A, B, C, etc.

%--------------------------------------
% Puntos de extensión

% Comente la siguiente sección en caso de que no hayan puntos de extensión o relaciones de tipo extends.
\subsection{Puntos de extensión}
\UCExtenssionPoint{
	% Cuando se dá la extensión del Caso de uso:
	El cliente quiere agregar detalles de su mascota asociada a su reservación.
}{
	% Durante la región (en que pasos se puede dar la extensión):
	En el paso \ref{CU2_Reg_mascota}.
}{
	% Casos de uso a los que extiende:
	\UCref{CU3}{Resgistrar Mascota}.
}
\UCExtenssionPoint{
	% Cuando se dá la extensión del Caso de uso:
	El cliente quiere pagar  o consultar el costo de la reservación.
}{
	% Durante la región (en que pasos se puede dar la extensión):
	En el paso \ref{CU2_Reg_mascota}.
}{
	% Casos de uso a los que extiende:
	\UCref{CUZ}{Resgistrar Pago}.
}

%-------------------------------------- TERMINA descripción del caso de uso.