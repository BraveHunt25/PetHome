%!TEX root = ../proyecto.tex

% Plantilla para caso de uso sencillo con ejemplos de comandos e intrucciones.
%-------------------------------------- COMIENZA descripción del caso de uso.

%\begin{UseCase}[archivo de imágen]{UCX}{Nombre del Caso de uso}{
%--------------------------------------
\begin{UseCase}{CU6}{Consultar reseñas de servicio }{
	Cuando el cliente quiera consultar las reseñas de su autoría que ha otorgado a los responsables de los servicios que haya adquirido con anterioridad en sus hospedajes, deberá ingresar a la pantalla de consulta de reseñas mediante un botón en la pantalla de inicio con el nombre de "Reseñas de servicio". Una vez dentro, podrá visualizar todas las reseñas dadas.
}
	\UCitem{Versión}{\color{Gray}
		0.1
	}\UCitem{Autor}{\color{Gray}
		Hernández Jiménez Erick Yael.
	}\UCitem{Supervisa}{\color{Gray}
		Ruiz Evaristo Marco Antonio
	}\UCitem{Cliente}{
		\hyperlink{Cliente}{Cliente} % No olvide dar de alta el actor.
	}\UCitem{Propósito}{\begin{Titemize}%Indique los fines, objetivos, propósitos o valores agregados del Caso de uso.
		\Titem Visualizar todas las reseñas dadas a los responsables de servicio.
	\end{Titemize}
	}\UCitem{Entradas}{\begin{Titemize}
		\Titem Ninguna.
		\end{Titemize}
	}\UCitem{Origen}{\begin{Titemize}
		\Titem Ninguno.
	\end{Titemize}
	}\UCitem{Salidas}{\begin{Titemize}
		% TODO: Dar de alta las entidades que se listan.
		\Titem \hyperlink{Entidad.Atributo}{ID de la cita}.
		\Titem \hyperlink{Entidad.Atributo}{Tipo de servicio}.
		\Titem \hyperlink{Empleados.Atributo}{Nombres}
		\Titem \hyperlink{Empleados.Atributo}{Primer apellido}
		\Titem \hyperlink{Empleados.Atributo}{Segundo apellido}
		\Titem \hyperlink{Entidad.Atributo}{ID del hospedaje}.
		\Titem \hyperlink{Entidad.Atributo}{Descripción}.
		\Titem \hyperlink{Entidad.Atributo}{Calificación}.
		\Titem Mensajes de error.
	\end{Titemize}
	}\UCitem{Destino}{\begin{Titemize}
		\Titem Se muestra en la pantalla \IUref{IUX}{Reseñas de servicio}.
	\end{Titemize}
	}\UCitem{Precondiciones}{\begin{Titemize}
		\Titem El cliente ya debe haber iniciado sesión.
		\Titem El cliente ya debe haber hecho al menos una reservación.
		\Titem El cliente ya debe tener al menos un servicio atendido durante su hospedaje.
		\Titem El cliente ya debe haber pagado en su totalidad su hospedaje.
	\end{Titemize}
	}\UCitem{Postcondiciones}{\begin{Titemize}
		\Titem Ninguna.
	\end{Titemize}
	}\UCitem{Errores}{\begin{Titemize}
		\Titem {\bf \hypertarget{CU6.E1}{E1}}: El cliente ingresa a la pantalla sin haber iniciado sesión.
		\Titem {\bf \hypertarget{CU6.E2}{E2}}: El cliente no ha hecho ninguna reservación.
		\Titem {\bf \hypertarget{CU6.E3}{E3}}: El cliente no tiene ningún servicio atendido.
		\Titem {\bf \hypertarget{CU6.E4}{E4}}: El cliente no ha pagado su servicio en su totalidad.
	\end{Titemize}
	}\UCitem{Tipo}{
		Caso de uso primario
	}\UCitem{Observaciones}{
		Ninguna.
	}
\end{UseCase}

%--------------------------------------
\begin{UCtrayectoria}
	\UCpaso[] El cliente indica al sistema que desea visualizar las reseñas consultadas presionando el botón \IUbutton{Reseñas de servicios}.
	\UCpaso[] El sistema verifica que el cliente haya iniciado sesión correctamente \ErrorRef{CU6}{E1}{No se ha iniciado sesión}\Trayref{CU6}{A}.
	\UCpaso[] El sistema verifica que ya haya hospedajes registrados \ErrorRef{UC6}{E2}{No se ha hecho ninguna reservación}\Trayref{CU6}{B}.
	\UCpaso[] El sistema verifica que tenga al menos un servicio ya atendido \ErrorRef{UC6}{E3}{No se ha contratado ningún servicio}\Trayref{CU6}{C}.
	\UCpaso[] El sistema verifica que se haya pagado en su totalidad al menos un servicio \ErrorRef{UC6}{E4}{No se han completado los pagos necesarios}\Trayref{CU6}{D}.
	\UCpaso[] Muestra la pantalla \IUref{IU1}{Reseñas de servicios} con las reseñas realizadas.
\end{UCtrayectoria}

%--------------------------------------
% Las trayectorias alternativas se identifican con Letras: A, B, C, etc.
\begin{UCtrayectoriaA}{CU6}{A}{El cliente no ha iniciado sesión}
	\UCpaso[] El sistema redirige al cliente a la pantalla \IUref{IU2}{Iniciar sesión} con el mensaje \MSGref{MSG-001}{Para visualizar tus reseñas, inicia sesión}.
\end{UCtrayectoriaA}
\begin{UCtrayectoriaA}{CU6}{B}{El cliente no ha hecho ninguna reservación}
	\UCpaso[] El sistema redirige al cliente a la pantalla \IUref{IU3}{Principal} con el mensaje \MSGref{MSG-002}{No existen hospedajes registrados}.
\end{UCtrayectoriaA}
\begin{UCtrayectoriaA}{CU6}{C}{El cliente no ha contratado al menos un servicio en un hospedaje}
	\UCpaso[] El sistema redirige al cliente a la pantalla \IUref{IU3}{Principal} con el mensaje \MSGref{MSG-003}{No existen servicios calificados}.
\end{UCtrayectoriaA}
\begin{UCtrayectoriaA}{CU6}{D}{El cliente no ha completado el pago de los servicios en su totalidad}
	\UCpaso[] El sistema redirige al cliente a la pantalla \IUref{IU3}{Principal} con el mensaje \MSGref{MSG-004}{No se han completado los pagos necesarios}.
\end{UCtrayectoriaA}

%--------------------------------------
% Puntos de extensión

% Comente la siguiente sección en caso de que no hayan puntos de extensión o relaciones de tipo extends.
\subsection{Puntos de extensión}
\UCExtenssionPoint{
	% Cuando se dá la extensión del Caso de uso:
	El usuario no recuerda cual es su contraseña o sospecha que su usuario está bloqueado.
}{
	% Durante la región (en que pasos se puede dar la extensión):
	Del paso \ref{CUX.etiqueta} al paso \ref{CUX.etiqueta}.
}{
	% Casos de uso a los que extiende:
	\UCref{CUZ}{Nombre del caso de uso}.
}
		
		
		
%-------------------------------------- TERMINA descripción del caso de uso.