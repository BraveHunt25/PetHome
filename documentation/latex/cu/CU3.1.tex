%!TEX root = ../proyecto.tex

% Plantilla para caso de uso sencillo con ejemplos de comandos e intrucciones.
%-------------------------------------- COMIENZA descripción del caso de uso.

%\begin{UseCase}[archivo de imágen]{UCX}{Nombre del Caso de uso}{
%--------------------------------------
\begin{UseCase}{CU3}{Editar Reservacion}{
	% La descripción debe describir el evento de inicio del caso de uso, Breve descripción de la trayectoria y estado final del caso de uso.
	Si un cliente olvido agregar alguna mascota o se equivoco en algun dato de su rservacion debe poder entrar al registro de la mascota y 		editar dicho registro, una vez editado la reservacion se actualizara 
}
	\UCitem{Versión}{\color{Gray}
		0.1	% Ponga un número de versión, 
	}\UCitem{Autor}{\color{Gray}
		Ruiz Evaristo Marco Antonio. % Analista responsable de especificar el CU
	}\UCitem{Supervisa}{\color{Gray}
		Hernández Jiménez Erick Yael. % Analista responsable de verificar que está correcto.
	% TODO: Dar de alta al actor Usuario
	}\UCitem{Actor}{
		\hyperlink{Cliente}{Cliente},
		\hyperlink{Recepcionista}{Recepsionista}
	}\UCitem{Propósito}{\begin{Titemize}%Indique los fines, objetivos, propósitos o valores agregados del Caso de uso.
		\Titem Poder corregir algun dato de el regitro de la mascota.
		\Titem Garantizar la informacion de la mascota.
		\Titem Controlar reservaciones.
	\end{Titemize}
	}\UCitem{Entradas}{\begin{Titemize}
		% TODO: Dar de alta las entidades que se listan.
			\Titem \hyperlink{mascota.nombre}{Nombre}. % El identificador no acepta acentos, espacios ni eñes.
			\Titem \hyperlink{mascota.tamano}{Tamaño}. % Liste todos los datos de entrada
			\Titem \hyperlink{mascota.fechaNacimiento}{Fecha de nacimiento}. 
			\Titem \hyperlink{mascota.RUAC}{RUAC}.
			\Titem \hyperlink{mascota.especie}{Especie}.
			\Titem \hyperlink{mascota.raza}{Raza}.
			\Titem \hyperlink{mascota.color}{Color}.
			\Titem \hyperlink{mascota.vacunas}{Vacunas}.
			\Titem \hyperlink{mascotaExtra.alergias}{Alergias}.
			\Titem \hyperlink{mascotaExtra.medicamentos}{Medicamentos}.
			\Titem \hyperlink{mascotaExtra.fechaUltimaDosis}{Ultima dosis administrada}.
			\Titem \hyperlink{mascotaExtra.dosis}{Dosis aplicadas}.
			\Titem \hyperlink{mascotaExtra.condicionesEspeciales}{Condiciones especiales}.
			\Titem \hyperlink{mascotaExtra.comentariosExtra}{Comentarios extra}.
			\Titem \hyperlink{mascotaExtra.comentariosExtra}{Comentarios extra}.
			\Titem \hyperlink{reservacion.fechaLlegada}{Fecha de llegada}.
			\Titem \hyperlink{reservacion.fechaSalida}{Fecha de Salida}.
			\Titem \hyperlink{reservacion.Sucursal}{Sucursal}.
		\end{Titemize}
	}\UCitem{Origen}{\begin{Titemize}
		\Titem Se introducen desde el teclado. % Indique por que medio se introducen los datos, 
		\Titem Se seleccionan de una lista con el mouse.       % Si es ḿas de uno indique que datos corresponden en cada medio de entrada.
	\end{Titemize}
	}\UCitem{Salidas}{\begin{Titemize}
		% TODO: Dar de alta las entidades que se listan.
		\Titem Mensajes de error. % Indique por que medio se introducen los datos, 
	\end{Titemize}
	}\UCitem{Destino}{\begin{Titemize}
		\Titem Se muestra en la pantalla \IUref{IU3}{Pantalla de Home Cliente}
		\Titem Se muestra en la pantalla \IUref{IU2}{Pantalla de Datos Obligatorios de la Mascota}
		\Titem Se muestra en la pantalla \IUref{IU4}{Pantalla de Datos de la Reservacion}
		\Titem Se muestra en la pantalla \IUref{IU3}{Pantalla de Datos Extra de la Mascota} % Indique por que medio se muestran los datos, 
		\Titem otros.          % Si es ḿas de uno indique que datos corresponden en cada medio de entrada.
	\end{Titemize}
	}\UCitem{Precondiciones}{\begin{Titemize}
		% Incluya Precondiciones lógicas, de negocio e incluso las que debe atender el usuario. 
		% Muchas precondiciones provienen de reglas de negocios, otras estarán asociadas a manejo de errores
		% Otras están relacionadas con casos de uso que deben ejecutarse previamente, como registrar un producto.
		\Titem El cliente debe tener una cuenta
		\Titem El cliente debe tener registrada una reservación.
		\Titem No debe haber otra mascota registrada con el mismo RUAC.
		\Titem La mascota debe contar con al menos las 3 vacunas solicitadas de acuerdo a \hyperlink{BR02}{BR02 Vacunas Obligatorias para la Mascota}
	\end{Titemize}
	}\UCitem{Postcondiciones}{\begin{Titemize}
		% Indique todas las postcondiciones
		% por ejemplo, Cambios en el sistema
		% Cambios en la BD una vez terminado el CU
		% Efectos colaterales
		% Condiciones de término.
		\Titem Se actualizaran los datos de la reservación.
	\end{Titemize}
	}\UCitem{Errores}{\begin{Titemize}
		% Escriba todos los errores que puedan ocurrir en el sistema, para cada error recuerde:
		% Punerle un identificador
		% Describir la condición o escenario que detona el error
		% Describa la forma en que debe reaccionar el sitema: si la reaccion corresponde a varios pasos use mejor una trayectoria alternativa.
		% Relacione el error con la trayectoria principal.
		\Titem {\bf \hypertarget{CU2.E1}{E1}}: Cuando el cliente no haya llenado algun campo obligatorio, el sistema muestra el mensaje \MSGref{MSG-001}{Campo obligatorio}.
			\Titem {\bf \hypertarget{CU2.E1}{E1}}: Cuando el RUAC de la mascota ya este registrado en el sistema, el sistema muestra el mensaje \MSGref{MSG-003}{RUAC ya existe}. 
			\Titem {\bf \hypertarget{CU2.E1}{E2}}: Cuando la mascota no cumpla con las vacunas obligatorias, el sistema muestra el mensaje \MSGref{MSG-002}{Vacunas obligatorias}. 
	\end{Titemize}
	}\UCitem{Tipo}{
		% Especifique el tipo de caso de us, puede ser: "Caso de uso primario" o 
		% "Viene de \\hyperref{CUY}{CUY nombre del CU}" cuando se desprende desde otro caso de uso mediante un extends.
		Para ejecutar este caso de uso debe pasar primero por el \UCref{CUX}{Iniciar Sesion}. o \UCref{CU1}{Registrar Cliente} y \UCref{CU2}{Registrar Mascota}
	}\UCitem{Observaciones}{
		% Indique las observaciones al caso de uso, las cuales pueden ser:
		- Ninguna
		% - Dudas sobre el procedimiento o la especificación.
		% - Issues detectados
		% - Suposiciones realizadas.
		% - Cualquier otra especificacion que considere pertinente que no pudo colocarse en los demás atributos del Caso de uso
		% - Aclaraciones.
		% - Notas para el usuario o desarrollador.
		% - Pendientes (TODO's) en caso de no usar los comentarios.
	}
\end{UseCase}

%--------------------------------------
\begin{UCtrayectoria}
	% Cada paso debe inicair con un Verbo en infinitivo, siempre especificando el objetivo del paso mas la accion en concreto.
	% \UCpaso[\UCactor] se refiere al actor y \UCpaso se refiere al sistema.
	% A continuación viene ejemplos de pasos:
	% En el siguiente paso: "Ingresa al sistema" es el objetivo del paso y "escribiendo la URL de la aplicación" es la acción en concreto.
	\UCpaso[] El cliente indica al sistema que desea editar los datos de la reservacion haciendo click en el botón \IUbutton{Editar}. en la pantalla \IUref{UI5}{Pantalla de Home Cliente}.
	\UCpaso []El sistema solicita los datos obligatorios de la mascota mediante la pantalla \IUref{IU2}{Pantalla de Datos Obligatorios de la Mascota}
	% En el siguinte paso se referencia una Interfaz:
	\UCpaso []El cliente proporciona los datos obligatorios que desea modificar de su mascota.
	\UCpaso []El cliente da click en el botón \IUbutton{Siguiente}.
	% En el siguiente paso se usa el comando \IUbutton 
	\UCpaso []El sistema verifica que los datos obligatorios de la mascota hayan sido llenados.
	\UCpaso []El sistema pasa a la pantalla \IUref{IU3}{Pantalla de Datos Extra de la Mascota}
	% En el siguiente paso se referencía un error.
	\UCpaso []El cliente da click en el botón \IUbutton{Siguiente}.
	\UCpaso []El sistema pasa a la pantalla \IUref{IU4}{Pantalla de Datos de la Reservacion}
	\UCpaso []El sistema proporciona el costo de la reservacion.
	\UCpaso []El cliente da click en el botón \IUbutton{Reservar}.
	\UCpaso []El sistema registra la reservación
\end{UCtrayectoria}


%--------------------------------------
% Las trayectorias alternativas se identifican con Letras: A, B, C, etc.
\begin{UCtrayectoriaA}{CU3}{A}{Si el cliente quiere modificar los datos extra de la mascota}
	\UCpaso[] El cliente indica al sistema que desea editar los datos de la reservacion haciendo click en el botón \IUbutton{Editar}. en la pantalla \IUref{UI5}{Pantalla de Home Cliente}.
	\UCpaso []El sistema pasa a la pantalla \IUref{IU2}{Pantalla de Datos Obligatorios de la Mascota}
	\UCpaso []El cliente da click en el botón \IUbutton{Siguiente}.
	% En el siguiente paso se usa el comando \IUbutton 
	\UCpaso []El sistema pasa a la pantalla \IUref{IU3}{Pantalla de Datos Extra de la Mascota}
	\UCpaso []El sistema solicita los datos extra de la mascota.
	\UCpaso []El cliente proporciona los datos extra que desea modificar de su mascota.
	% En el siguiente paso se referencía un error.
	\UCpaso []El cliente da click en el botón \IUbutton{Siguiente}.
	\UCpaso []El sistema pasa a la pantalla \IUref{IU4}{Pantalla de Datos de la Reservacion}
	\UCpaso []El sistema proporciona el costo de la reservacion.
	\UCpaso []El cliente da click en el botón \IUbutton{Reservar}.
	\UCpaso []El sistema registra la reservación.
\end{UCtrayectoriaA}

\begin{UCtrayectoriaA}{CU3}{B}{Si el cliente quiere agregar otra mascota en esa reservacion}
	\UCpaso[] El cliente indica al sistema que desea editar los datos de la reservacion haciendo click en el botón \IUbutton{Editar}. en la pantalla \IUref{UI5}{Pantalla de Home Cliente}.
	\UCpaso []El sistema pasa a la pantalla \IUref{IU2}{Pantalla de Datos Obligatorios de la Mascota}
	\UCpaso []El cliente da click en el botón \IUbutton{Siguiente}.
	\UCpaso []El sistema pasa a la pantalla \IUref{IU3}{Pantalla de Datos Extra de la Mascota}
	\UCpaso []El cliente da click en el boton \IUbutton{Agregar Mascota}.
	\UCpaso []El sistema despliega otra seccion para agregar otra mascota 
	\UCpaso []El cliente proporciona los datos de su otra mascota.
	\UCpaso []El cliente da click en el botón \IUbutton{Siguiente}.
	\UCpaso []El sistema pasa a la pantalla \IUref{IU4}{Pantalla de Datos de la Reservacion}
	\UCpaso []El sistema proporciona el costo de la reservacion.
	\UCpaso []El cliente da click en el botón \IUbutton{Reservar}.
	\UCpaso []El sistema registra la reservación
	\UCpaso []El sistema registra la mascota.
\end{UCtrayectoriaA}

%\begin{UCtrayectoriaA}{CU3}{C}{Si el cliente quiere editar un servicio}
%	\UCpaso[] El cliente indica al sistema que desea editar los datos de la reservacion haciendo click en el botón \IUbutton{Editar}. en la pantalla \IUref{UI5}{Pantalla de Home Cliente}.
%	\UCpaso []El sistema pasa a la pantalla \IUref{IU2}{Pantalla de Datos Obligatorios de la Mascota}
%	\UCpaso []El cliente da click en el botón \IUbutton{Siguiente}.
%	\UCpaso []El sistema pasa a la pantalla \IUref{IU3}{Pantalla de Datos Extra de la Mascota}
%	\UCpaso []El cliente da click en el botón \IUbutton{Siguiente}.
%	\UCpaso []El sistema pasa a la pantalla \IUref{IU4}{Pantalla de Datos de la Reservacion}
%	\UCpaso []El cliente da click en el botón \IUbutton{Agregar servicio}.
%	\UCpaso []El sistema despliega los servicios con los que cuenta el hotel.	
%	\UCpaso []El cliente selecciona el servicio que desea agregar o quitar.
%	\UCpaso []El sistema proporciona el costo de la reservacion.
%	\UCpaso []El cliente da click en el botón \IUbutton{Reservar}.
%	\UCpaso []El sistema registra la reservación.

%\end{UCtrayectoriaA}





%--------------------------------------
% Puntos de extensión

% Comente la siguiente sección en caso de que no hayan puntos de extensión o relaciones de tipo extends.
%\subsection{Puntos de extensión}
%\UCExtenssionPoint{
	% Cuando se dá la extensión del Caso de uso:
%	El usuario no recuerda cual es su contraseña o sospecha que su usuario está bloqueado.
%}{
	% Durante la región (en que pasos se puede dar la extensión):
%	Del paso \ref{CUX.etiqueta} al paso \ref{CUX.etiqueta}.
%}{
	% Casos de uso a los que extiende:
%	\UCref{CUZ}{Nombre del caso de uso}.
%}
		
		
		
%-------------------------------------- TERMINA descripción del caso de uso.