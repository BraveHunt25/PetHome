%!TEX root = ../proyecto.tex

% Plantilla para caso de uso sencillo con ejemplos de comandos e intrucciones.
%-------------------------------------- COMIENZA descripción del caso de uso.

%\begin{UseCase}[archivo de imágen]{UCX}{Nombre del Caso de uso}{
%--------------------------------------
\begin{UseCase}{CU5}{Log-out}{
	Cuando un cliente a terminado de usar su sesion debe poder cerrar dicha sesion para que el cliente pueda mantener su sesion segura.
}
	\UCitem{Versión}{\color{Gray}
		0.1	% Ponga un número de versión, 
	}\UCitem{Autor}{\color{Gray}
		Ruiz Evaristo Marco Antonio. % Analista responsable de especificar el CU
	}\UCitem{Supervisa}{\color{Gray}
		Hernández Jiménez Erick Yael. % Analista responsable de verificar que está correcto.
	% TODO: Dar de alta al actor Usuario
	}\UCitem{Actor}{
		\hyperlink{Cliente}{Cliente},
		 % No olvide dar de alta el actor.
	}\UCitem{Propósito}{\begin{Titemize}%Indique los fines, objetivos, propósitos o valores agregados del Caso de uso.
		\Titem Proteger los datos del cliente.
		\Titem Mantener seguro el sistema.
	\end{Titemize}
	}\UCitem{Entradas}{\begin{Titemize}
		% TODO: Dar de alta las entidades que se listan.
		\Titem Ninguna
		\end{Titemize}
	}\UCitem{Origen}{\begin{Titemize}
		\Titem Se da click desde el mouse.     % Si es ḿas de uno indique que datos corresponden en cada medio de entrada.
	\end{Titemize}
	}\UCitem{Salidas}{\begin{Titemize}
		% TODO: Dar de alta las entidades que se listan.
		\Titem Ninguna
	\end{Titemize}
	}\UCitem{Destino}{\begin{Titemize}
		\Titem Se muestra en la pantalla \IUref{IU3}{Pantalla de Home Cliente}
	\end{Titemize}
	}\UCitem{Precondiciones}{\begin{Titemize}
		% Incluya Precondiciones lógicas, de negocio e incluso las que debe atender el usuario. 
		% Muchas precondiciones provienen de reglas de negocios, otras estarán asociadas a manejo de errores
		% Otras están relacionadas con casos de uso que deben ejecutarse previamente, como registrar un producto.
		\Titem El cliente debe tener una cuenta
	\end{Titemize}
	}\UCitem{Postcondiciones}{\begin{Titemize}
		% Indique todas las postcondiciones
		% por ejemplo, Cambios en el sistema
		% Cambios en la BD una vez terminado el CU
		% Efectos colaterales
		% Condiciones de término.
		\Titem Se cierra la sesion del cliente.
	\end{Titemize}
	}\UCitem{Errores}{\begin{Titemize}
		% Escriba todos los errores que puedan ocurrir en el sistema, para cada error recuerde:
		% Punerle un identificador
		% Describir la condición o escenario que detona el error
		% Describa la forma en que debe reaccionar el sitema: si la reaccion corresponde a varios pasos use mejor una trayectoria alternativa.
		% Relacione el error con la trayectoria principal.
		\Titem Ninguno
	\end{Titemize}
	}\UCitem{Tipo}{
		% Especifique el tipo de caso de us, puede ser: "Caso de uso primario" o 
		% "Viene de \\hyperref{CUY}{CUY nombre del CU}" cuando se desprende desde otro caso de uso mediante un extends.
		Para ejecutar este caso de uso debe pasar primero por el \UCref{CUX}{Iniciar Sesion}. o \UCref{CU1}{Registrar Cliente}
	}\UCitem{Observaciones}{
		% Indique las observaciones al caso de uso, las cuales pueden ser:
		% - Ninguna
		 Dudas sobre el procedimiento o la especificación.
		% - Issues detectados
		% - Suposiciones realizadas.
		- Falta definir los actores
		% - Aclaraciones.
		% - Notas para el usuario o desarrollador.
		% - Pendientes (TODO's) en caso de no usar los comentarios.
	}
\end{UseCase}

%--------------------------------------
\begin{UCtrayectoria}
	% Cada paso debe inicair con un Verbo en infinitivo, siempre especificando el objetivo del paso mas la accion en concreto.
	% \UCpaso[\UCactor] se refiere al actor y \UCpaso se refiere al sistema.
	% A continuación viene ejemplos de pasos:
	% En el siguiente paso: "Ingresa al sistema" es el objetivo del paso y "escribiendo la URL de la aplicación" es la acción en concreto.
	\UCpaso[] El cliente indica al sistema que desea cerrar sesion haciendo click en el botón \IUbutton{Salir}. en la pantalla \IUref{UI5}{Pantalla de Home Cliente}.
	\UCpaso []El sistema cierra la sesión. 
\end{UCtrayectoria}


%--------------------------------------
% Las trayectorias alternativas se identifican con Letras: A, B, C, etc.
%\begin{UCtrayectoriaA}{CUX}{LETRA}{Condición que hace que se ejecute esta trayectoria}
%	\UCpaso Especifique los pasos  de la trayectoria.
	% Se puede desprender otra trayectria alternativa si es necesario.
	% Finalice la trayectoria indicando si la ejecución se integra a la trayectoria anterior o si termina la ejecución del CU.
	% Verifique que la redacción de la trayectoria deje en claro si el objetivo del CU se alcanzó o no.
%	\UCpaso[] El Caso de Uso continúa en el paso \ref{UCX.introduceDatos}.
%\end{UCtrayectoriaA}


%--------------------------------------
% Puntos de extensión

% Comente la siguiente sección en caso de que no hayan puntos de extensión o relaciones de tipo extends.
%\subsection{Puntos de extensión}
%\UCExtenssionPoint{
	% Cuando se dá la extensión del Caso de uso:
%	El usuario no recuerda cual es su contraseña o sospecha que su usuario está bloqueado.
%}{
	% Durante la región (en que pasos se puede dar la extensión):
%	Del paso \ref{CUX.etiqueta} al paso \ref{CUX.etiqueta}.
%}{
	% Casos de uso a los que extiende:
	%\UCref{CUZ}{Nombre del caso de uso}.
%}
		
		
		
%-------------------------------------- TERMINA descripción del caso de uso.