%!TEX root = ../proyecto.tex

% Plantilla para caso de uso sencillo con ejemplos de comandos e intrucciones.
%-------------------------------------- COMIENZA descripción del caso de uso.

%\begin{UseCase}[archivo de imágen]{UCX}{Nombre del Caso de uso}{
%--------------------------------------
\begin{UseCase}{CU7}{Alta de servicio}{
	Cuando el cliente desée agregar servicios adicionales al hospedaje de su mascota, deberá ingresar a la pantalla de "Agregar servicio" mediante el botón homónomo en la pantalla de reservación. Una vez dentro, podrá introducir los datos del servicio que desee agregar.
}
	\UCitem{Versión}{\color{Gray}
		0.1
	}\UCitem{Autor}{\color{Gray}
		Hernández Jiménez Erick Yael.
	}\UCitem{Supervisa}{\color{Gray}
		Ruiz Evaristo Marco Antonio.
	}\UCitem{Cliente}{
		\hyperlink{Cliente}{Cliente} 
	}\UCitem{Propósito}{\begin{Titemize}
		\Titem Que el cliente pueda contratar servicios para la mascota antes del hospedaje 
	\end{Titemize}
	}\UCitem{Entradas}{\begin{Titemize}
		\Titem \hyperlink{CATALOGO_SERVICIOS.Nombre_servicio}{Nombre del servicio}.
		\Titem \hyperlink{CITA.Fecha}{Fecha del servicio}.
		\Titem \hyperlink{EMPLEADO.Nombres}{Nombres}.
		\Titem \hyperlink{EMPLEADO.Primer_apellido}{Primer apellido}.
		\Titem \hyperlink{EMPLEADO.Segundo_apellido}{Segundo apellido}.
		\end{Titemize}
	}\UCitem{Origen}{\begin{Titemize}
		\Titem \hyperlink{CATALOGO_SERVICIOS.Nombre_servicio}{Nombre del servicio}: Seleccionado con el mouse desde un menú desplegable.
		\Titem \hyperlink{CITA.Fecha}{Fecha del servicio}: Seleccionado con el mouse desde el campo de selección.
		\Titem \hyperlink{EMPLEADO.Nombres}{Nombres}: Seleccionado con el mouse desde un menú desplegable.
		\Titem \hyperlink{EMPLEADO.Primer_apellido}{Primer apellido}: Seleccionado con el mouse desde un menú desplegable.
		\Titem \hyperlink{EMPLEADO.Segundo_apellido}{Segundo apellido}: Seleccionado con el mouse desde un menú desplegable.
	\end{Titemize}
	}\UCitem{Salidas}{\begin{Titemize}
		\Titem \hyperlink{CATALOGO_SERVICIOS.Nombre_servicio}{Nombre del servicio}.
		\Titem \hyperlink{CITA.Fecha}{Fecha del servicio}.
		\Titem \hyperlink{HOSPEDAJE.ID_Reservacion}{ID hospedaje}.
		\Titem \hyperlink{CATALOGO_SERVICIOS.Costo_base}{Costo del servicio}.
		\Titem \hyperlink{EMPLEADO.Nombres}{Nombres}.
		\Titem \hyperlink{EMPLEADO.Primer_apellido}{Primer apellido}.
		\Titem \hyperlink{EMPLEADO.Segundo_apellido}{Segundo apellido}.
		\Titem \hyperlink{CITAS.ID_Cita}{ID de la cita}.
		\Titem Mensajes de error.
	\end{Titemize}
	}\UCitem{Destino}{\begin{Titemize}
		\Titem \hyperlink{CATALOGO_SERVICIOS.Nombre_servicio}{Nombre del servicio}: Se muestra en la pantalla \IUref{IUX}{Reservación} y en la pantalla \IUref{IU7}{Alta de servicio}.
		\Titem \hyperlink{CITA.Fecha}{Fecha del servicio}: Se muestra en la pantalla \IUref{IUX}{Reservación} y en la pantalla \IUref{IU7}{Alta de servicio}.
		\Titem \hyperlink{HOSPEDAJE.ID_Reservacion}{ID hospedaje}: Se muestra en la pantalla \IUref{IUX}{Reservación}.
		\Titem \hyperlink{CATALOGO_SERVICIOS.Costo_base}{Costo del servicio}: Se muestra en la pantalla \IUref{IU7}{Alta de servicio}.
		\Titem \hyperlink{EMPLEADO.Nombres}{Nombres}: Se muestra en la pantalla \IUref{IUX}{Reservación} y en la pantalla \IUref{IU7}{Alta de servicio}.
		\Titem \hyperlink{EMPLEADO.Primer_apellido}{Primer apellido}: Se muestra en la pantalla \IUref{IUX}{Reservación} y en la pantalla \IUref{IU7}{Alta de servicio}.
		\Titem \hyperlink{EMPLEADO.Segundo_apellido}{Segundo apellido}: Se muestra en la pantalla \IUref{IUX}{Reservación} y en la pantalla \IUref{IU7}{Alta de servicio}.
		\Titem \hyperlink{CITAS.ID_Cita}{ID de la cita}: Se muestra en la pantalla \IUref{IUX}{Reservación}.
	\end{Titemize}
	}\UCitem{Precondiciones}{\begin{Titemize}
		\Titem El cliente deberá llenar con anterioridad los datos de su reservación en las pantallas correspondientes.
	\end{Titemize}
	}\UCitem{Postcondiciones}{\begin{Titemize}
		\Titem Se le asignará la cita al responsable de servicio.
		\Titem Se registrará la cita al hospedaje de la mascota.
	\end{Titemize}
	}\UCitem{Errores}{\begin{Titemize}
		\Titem {\bf \hypertarget{CU7.E1}{E1}}: El cliente no ha rellenado los campos necesarios para la reservación. 
		\Titem {\bf \hypertarget{CU7.E2}{E2}}: No existe el servicio en la sucursal solicitada. 
		\Titem {\bf \hypertarget{CU7.E3}{E3}}: El cliente solicita el servicio para una fecha fuera del lapso de la estancia del cliente.
		\Titem {\bf \hypertarget{CU7.E4}{E4}}: No hay responsables disponibles en la fecha solicitada. 
		\Titem {\bf \hypertarget{CU7.E5}{E5}}: El responsable de servicio no tiene disponibilidad para la fecha asignada.
	\end{Titemize}
	}\UCitem{Tipo}{
		Viene de \hyperlink{CUY}{CUY Reservación}.
	}\UCitem{Observaciones}{\begin{Titemize}
		\Titem Se hace la suposición de que la única etapa para cambiar citas de servicios es antes de la fecha del hospedaje.
		\Titem Se hace la consideración de que la disponibilidad de los responsables no varía durante el tiempo de agendado.
	\end{Titemize}
	}
\end{UseCase}

%--------------------------------------
\begin{UCtrayectoria}
	\UCpaso[] El cliente indica al sistema que desea registrar un servicio adicional para el hospedaje de su mascota con el botón \IUbutton{Agregar servicio} desde la pantalla \IUref{IUX}{Reservación}.
	\UCpaso[] \label{UC7.IntroduceDatosCita} El sistema solicita los datos \hyperlink{CATALOGO_SERVICIOS.Nombre_servicio}{Servicio} y \hyperlink{CITA.Fecha}{Fecha del servicio} para contratar desde la pantalla \IUref{IU2}{Agregar servicio}.
	\UCpaso[] El cliente solicita al sistema buscar a los responsable del servicio solicitado disponibles con el botón \IUbutton{Checar disponibilidad}.
	\UCpaso[] El sistema verifica que el cliente haya llenado los campos  correspondientes a \hyperlink{CATALOGO_SERVICIOS.Nombre_servicio}{Nombre del servicio} y \hyperlink{CITA.Fecha}{Fecha del servicio} \ErrorRef{CU7}{E1}{No se han llenado los datos necesarios para realizar la cita} \Trayref{CU7}{A}.
	\UCpaso[] El sistema verifica que el servicio exista en la sucursal en la que se agenda \ErrorRef{CU7}{E2}{No existe este servicio en la sucursal} \Trayref{CU7}{B}.
	\UCpaso[] El sistema verifica que la fecha de la cita corresponda al lapso de la estancia \ErrorRef{CU7}{E4}{Fecha solicitada no válida} \Trayref{CU7}{C}.
	\UCpaso[] El sistema verifica que exista responsables disponibles en la fecha que se agenda \ErrorRef{CU7}{E5}{No hay responsables disponibles para el servicio solicitado} \Trayref{CU7}{D}.
	\UCpaso[] El sistema actualiza el menú desplegable de responsables de servicio en la pantalla \IUref{IUX}{Reservación} y el precio del servicio que selecciona.
	\UCpaso[] \label{UC7.IntroduceResponsableServicio} El cliente deja vacío el campo o selecciona al responsable de servicio que desea \Trayref{CU7}{E}
	\UCpaso[] El cliente solicita al sistema que le asigne un responsable de servicio mediante el botón \IUbutton{Agendar cita}.
	\UCpaso[] El sistema asigna un responsable de servicio disponible.
	\UCpaso[] El sistema redirige al cliente a la pantalla \IUref{IUX}{Reservación} con los datos de salida mostrados en el espacio correspondiente y el mensaje \MSGref{MSG-001}{Cita reservada con éxito}.
	\UCpaso[] El sistema habilita los botones \IUbutton{Cambiar servicio} y \IUbutton{Eliminar servicio} en la pantalla \IUref{IUX}{Reservación}.
\end{UCtrayectoria}

\begin{UCtrayectoriaA}{CU7}{A}{El cliente no ha llenado los datos necesarios para realizar la cita}
	\UCpaso El sistema le indica al cliente que faltan campos por llenar para contratar el servicio solicitado mediante el mensaje \MSGref{MSG-00X}{No se han rellenado los datos necesarios para contratar un servicio adicional}.
	\UCpaso El sistema regresa al cliente al paso \ref{UC7.IntroduceDatos}.
\end{UCtrayectoriaA}
\begin{UCtrayectoriaA}{CU7}{B}{El cliente solicita un servicio que no existe en la sucursal}
	\UCpaso El sistema le indica al cliente que no existe el servicio en la sucursal solicitada mediante el mensaje \MSGref{MSG-00X}{No hay responsables para este servicio en esta sucursal}.
	\UCpaso El sistema regresa al cliente al paso \ref{UC7.IntroduceDatos}.
\end{UCtrayectoriaA}
\begin{UCtrayectoriaA}{CU7}{C}{El cliente introduce una fecha para la cita fuera del lapso de la estancia para la que reserva}
	\UCpaso El sistema le indica al cliente que la fecha de la cita es solicitada fuera del periodo de estancia en el hotel mediante el mensaje \MSGref{MSG-00X}{La fecha solicitada no corresponde al tiempo de la estancia, seleccione otra fecha}.
	\UCpaso El sistema regresa al cliente al paso \ref{UC7.IntroduceDatos}.
\end{UCtrayectoriaA}
\begin{UCtrayectoriaA}{CU7}{D}{El cliente solicita un servicio en el cual no hay responsables disponible}
	\UCpaso El sistema le indica al cliente que no hay responsables para el servicio extra solicitado mediante el mensaje \MSGref{MSG-00X}{No hay responsables disponibles para el servicio solicitado}.
	\UCpaso El sistema regresa al cliente al paso \ref{UC7.IntroduceDatos}.
\end{UCtrayectoriaA}
\begin{UCtrayectoriaA}{CU7}{E}{El cliente selecciona algún responsable de servicio en el menú desplegable}
	\UCpaso[] El cliente solicita al sistema agendar su cita con el responsable marcado que le asigne uno mediante el botón \IUbutton{Agendar cita}.
	\UCpaso[] El sistema redirige al cliente a la pantalla \IUref{IUX}{Reservación} con los datos de salida mostrados en el espacio correspondiente y el mensaje \MSGref{MSG-001}{Cita reservada con éxito}.
	\UCpaso[] El sistema habilita los botones \IUbutton{Cambiar servicio} y \IUbutton{Eliminar servicio} en la pantalla \IUref{IUX}{Reservación}.
\end{UCtrayectoriaA}



%--------------------------------------
% Puntos de extensión

% Comente la siguiente sección en caso de que no hayan puntos de extensión o relaciones de tipo extends.
\subsection{Puntos de extensión}
\UCExtenssionPoint{
	% Cuando se dá la extensión del Caso de uso:
	El usuario no recuerda cual es su contraseña o sospecha que su usuario está bloqueado.
}{
	% Durante la región (en que pasos se puede dar la extensión):
	Del paso \ref{CUX.etiqueta} al paso \ref{CUX.etiqueta}.
}{
	% Casos de uso a los que extiende:
	\UCref{CUZ}{Nombre del caso de uso}.
}
		
		
		
%-------------------------------------- TERMINA descripción del caso de uso.