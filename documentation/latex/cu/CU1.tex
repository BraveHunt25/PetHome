%!TEX root = ../proyecto.tex

% Plantilla para caso de uso sencillo con ejemplos de comandos e intrucciones.
%-------------------------------------- COMIENZA descripción del caso de uso.

%\begin{UseCase}[archivo de imágen]{UCX}{Nombre del Caso de uso}{
%--------------------------------------
\begin{UseCase}{CU1}{Registrar Cliente}{
	Cuando un cliente quiere reservar en el hotel, se debe dar de alta en el sistema con sus datos personales, una vez registrado podrá ingresar al sistema.
}
	\UCitem{Versión}{\color{Gray}
		0.2	% Ponga un número de versión, 
	}\UCitem{Autor}{\color{Gray}
		Ruiz Evaristo Marco Antonio. % Analista responsable de especificar el CU
	}\UCitem{Supervisa}{\color{Gray}
		Hernández Jiménez Erick Yael. % Analista responsable de verificar que está correcto.
	% TODO: Dar de alta al actor Usuario
	}\UCitem{Actor}{
		\hyperlink{Cliente}{Cliente} % No olvide dar de alta el actor.
	}\UCitem{Propósito}{\begin{Titemize}%Indique los fines, objetivos, propósitos o valores agregados del Caso de uso.
		\Titem Permitir al cliente realizar reservaciones y registrar mascotas.
		\Titem Garantizar la privacidad de los datos del cliente y soportar la responsabilidad de los clientes en el sistema.
	\end{Titemize}
	}\UCitem{Entradas}{\begin{Titemize}
		% TODO: Dar de alta las entidades que se listan.
		\Titem \hyperlink{Cliente.nombre}{Nombre(s)} del cliente % El identificador no acepta acentos, espacios ni eñes.
		\Titem \hyperlink{Cliente.primerApellido}{Primer apellido} del cliente % Liste todos los datos de entrada
		\Titem \hyperlink{Cliente.segundoApellido}{Segundo apellido} del cliente. 
		\Titem \hyperlink{Cliente.CURP}{CURP } del cliente
		\Titem \hyperlink{Cliente.telefono}{Numero telefonico} del cliente.
		\Titem \hyperlink{Cliente.correo}{Correo electónico} del cliente
		\Titem \hyperlink{Cliente.direccion}{Ubicación origen} coordenadas latitud y longitud.
		\Titem \hyperlink{Cliente.Atributo}{Contraseña} con confirmación.
		\end{Titemize}
	}\UCitem{Origen}{\begin{Titemize}
		\Titem Se introducen desde el teclado y mouse.
		\Titem Ubicacion origen. La selecciona el usuario en el mapa % Indique por que medio se introducen los datos, 
	\end{Titemize}
	}\UCitem{Salidas}{\begin{Titemize}
		% TODO: Dar de alta las entidades que se listan.
		\Titem Mensajes de error. %Agregar errores explícitamente
		
	\end{Titemize}
	}\UCitem{Destino}{\begin{Titemize}
%		\Titem Se muestra en la pantalla \IUref{IU01}{}.. % Indique por que medio se muestran los datos, 
%		\Titem otros.          % Si es ḿas de uno indique que datos corresponden en cada medio de entrada.
		\Titem Se muestra en la pantalla \IUref{IU1}{Registrar cliente}.
	\end{Titemize}
	}\UCitem{Precondiciones}{\begin{Titemize}
		\Titem No debe haber otro cliente registrado con el mismo CURP ni correo electrónico.
	\end{Titemize}
	}\UCitem{Postcondiciones}{\begin{Titemize}
		\Titem Habrá un nuevo cliente registrado en el sistema.
		\Titem El cliente podrá registrar mascotas.
		\Titem El cliente podrá realizar reservaciones.
		\Titem El cliente podrá iniciar sesión.
		\Titem Se mostrara la pantalla \IUref{IU5}{Home cliente}
	\end{Titemize}
	}\UCitem{Errores}{\begin{Titemize}
		% Escriba todos los errores que puedan ocurrir en el sistema, para cada error recuerde:
		% Punerle un identificador
		% Describir la condición o escenario que detona el error
		% Describa la forma en que debe reaccionar el sitema: si la reaccion corresponde a varios pasos use mejor una trayectoria alternativa.
		% Relacione el error con la trayectoria principal.
		\Titem {\bf \hypertarget{CU1.E1}{E1}}: Cuando el cliente no haya llenado algun campo obligatorio, el sistema muestra el mensaje \MSGref{MSG-001}{Campo obligatorio} y regresa al paso \ref{UC1.paso3}.
			\Titem {\bf \hypertarget{CU1.E2}{E2}}: Cuando el CURP del cliente ya este registrado en el sistema, el sistema muestra el mensaje \MSGref{MSG-005}{CURP ya existente} y regresa al paso \ref{UC1.paso3}.
			\Titem {\bf \hypertarget{CU1.E3}{E3}}: Cuando el correo electronico del cliente ya este registrado en el sistema, el sistema muestra el mensaje \MSGref{MSG-006}{Correo electrónico ya existente} y regresa al paso \ref{UC1.paso3}
			\Titem {\bf \hypertarget{CU1.E4}{E4}}: Cuando el cliente ingrese un tipo de dato diferente al solicitado, el sistema muestra el mensaje \MSGref{MSG-007}{Dato no válido} y regresa al paso \ref{UC1.paso3}
			
	\end{Titemize}
	}\UCitem{Tipo}{
		% Especifique el tipo de caso de us, puede ser: "Caso de uso primario" o 
		% "Viene de \\hyperref{CUY}{CUY nombre del CU}" cuando se desprende desde otro caso de uso mediante un extends.
		Caso de uso primario
	}\UCitem{Observaciones}{
		% Indique las observaciones al caso de uso, las cuales pueden ser:
		%
		 Ninguna
		% - Dudas sobre el procedimiento o la especificación.
		% - Issues detectados
		% - Suposiciones realizadas.
		% - Cualquier otra especificacion que considere pertinente que no pudo colocarse en los demás atributos del Caso de uso
		% - Aclaraciones.
		% - Notas para el usuario o desarrollador.
		% - Pendientes (TODO's) en caso de no usar los comentarios.
	}
\end{UseCase}

%--------------------------------------
\begin{UCtrayectoria}
	% Cada paso debe inicair con un Verbo en infinitivo, siempre especificando el objetivo del paso mas la accion en concreto.
	% \UCpaso[\UCactor] se refiere al actor y \UCpaso se refiere al sistema.
	% A continuación viene ejemplos de pasos:
	% En el siguiente paso: "Ingresa al sistema" es el objetivo del paso y "escribiendo la URL de la aplicación" es la acción en concreto.
	\UCpaso[] \label{UC1.paso1}El cliente indica al sistema que desea registrarse presionando el botón \IUbutton{Registrar cliente} de la pantalla \IUref{IU3}{Pantalla de inicio}.
    \UCpaso[] \label{UC1.paso2}El sistema solicita los datos personales del cliente mediante la pantalla \IUref{IU1}{Registrar Cliente}.
    \UCpaso[] \label{UC1.paso3} El cliente proporciona sus datos personales.
    \UCpaso[] \label{UC1.paso4} El cliente solicita el registro mediante el botón \IUbutton{Registrar} de la pantalla \IUref{IU2}{Registrar Cliente}.
    \UCpaso[] El sistema verifica que todos los campos obligatorios se hayan llenado \ErrorRef{CU1}{E1}{Campo obligatorio}.
    \UCpaso[] El sistema verifica que los datos introducidos tengan un formato válido en los campos correspondientes \ErrorRef{CU1}{E4}{Dato no válido}.
    \UCpaso[] \label{UC1.paso5} El sistema verifica que no haya un cliente ya registrado con el mismo CURP \ErrorRef{CU1}{E2}{CURP ya existente}.
    \UCpaso[] \label{UC1.paso6} El sistema verifica que no haya un cliente ya registrado con el mismo correo electrónico \ErrorRef{CU1}{E3}{Correo electrónico ya existente}.
    \UCpaso[] \label{UC1.paso7} El sistema muestra la pantalla \IUref{IU5}{Home cliente} con el mensaje \MSGref{MSG-003}{No hay reservaciones}.
\end{UCtrayectoria}


%--------------------------------------
% Las trayectorias alternativas se identifican con Letras: A, B, C, etc.


%--------------------------------------
% Puntos de extensión

% Comente la siguiente sección en caso de que no hayan puntos de extensión o relaciones de tipo extends.
%\subsection{Puntos de extensión}
%\UCExtenssionPoint{
	% Cuando se dá la extensión del Caso de uso:
%	El usuario no recuerda cual es su contraseña o sospecha que su usuario está bloqueado.
%}{
	% Durante la región (en que pasos se puede dar la extensión):
%	Del paso \ref{CUX.etiqueta} al paso \ref{CUX.etiqueta}.
%}{
	% Casos de uso a los que extiende:
%	\UCref{CUZ}{Nombre del caso de uso}.
%}
		
		
		
%-------------------------------------- TERMINA descripción del caso de uso.