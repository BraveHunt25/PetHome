%!TEX root = ../proyecto.tex

% Plantilla para caso de uso sencillo con ejemplos de comandos e intrucciones.
%-------------------------------------- COMIENZA descripción del caso de uso.

%\begin{UseCase}[archivo de imágen]{UCX}{Nombre del Caso de uso}{
%--------------------------------------
\begin{UseCase}{CU3}{Registrar Mascota}{
		Cuando un cliente quiere realizar una reservacion en hotel, debe dar de alta en el sistema los datos de su mascota, una vez registrados continuara con la reservacion.
	}
	\UCitem{Versión}{\color{Gray}
		0.2	% Ponga un número de versión, 
	}\UCitem{Autor}{\color{Gray}
		Ruiz Evaristo Marco Antonio. % Analista responsable de especificar el CU
	}\UCitem{Supervisa}{\color{Gray}
		Hernández Jiménez Erick Yael. % Analista responsable de verificar que está correcto.
		% TODO: Dar de alta al actor Usuario
	}\UCitem{Actor}{
		\hyperlink{Cliente}{Cliente}% No olvide dar de alta el actor.
	}\UCitem{Propósito}{\begin{Titemize}%Indique los fines, objetivos, propósitos o valores agregados del Caso de uso.
		\Titem Facilitar al cliente la reservación de manera eficiente, ofreciendo un proceso intuitivo y simplificado para satisfacer sus necesidades.
		\Titem Garantizar la seguridad y privacidad de los datos de las mascotas, además de proporcionar un sistema confiable que soporte las necesidades de los clientes al manejar eficientemente sus requerimientos.
		\Titem Almacenar los datos de las mascotas de forma segura para permitir reservaciones futuras, mejorando la experiencia del cliente al proporcionar un proceso de reservación más rápido y personalizado.
		\end{Titemize}
	}\UCitem{Entradas}{\begin{Titemize}
			% TODO: Dar de alta las entidades que se listan.
			\Titem \hyperlink{mascota.nombre}{Nombre}. % El identificador no acepta acentos, espacios ni eñes.
			\Titem \hyperlink{mascota.tamano}{Tamaño}. % Liste todos los datos de entrada
			\Titem \hyperlink{mascota.fechaNacimiento}{Fecha de nacimiento}. 
			\Titem \hyperlink{mascota.RUAC}{RUAC}.
			\Titem \hyperlink{mascota.especie}{Especie}.
			\Titem \hyperlink{mascota.raza}{Raza}.
			\Titem \hyperlink{mascota.color}{Color}.
			\Titem \hyperlink{mascota.vacunasAplicadas}{Vacunas aplicadas}.
			\Titem \hyperlink{mascota.alergias}{Alergias}.
			\Titem \hyperlink{mascota.medicamentos}{Medicamentos}
			\Titem \hyperlink{mascota.comentariosExtra}{Comentarios extra}.
		\end{Titemize}
	}\UCitem{Origen}{\begin{Titemize}
			\Titem Se introducen desde el teclado. % Indique por que medio se introducen los datos, 
			\Titem Se seleccionan de un selector de fecha (calendario) con el mouse.
			\Titem Se marca un checkbox con el mouse.
		
		\end{Titemize}
	}\UCitem{Salidas}{\begin{Titemize}
			% TODO: Dar de alta las entidades que se listan.
			\Titem Mensajes de error. % Indique por que medio se introducen los datos, 
		\end{Titemize}
	}\UCitem{Destino}{\begin{Titemize}
			\Titem Se muestra en la pantalla \IUref{IU3}{Registrar Mascota}.
			\Titem Se muestra en la pantalla \IUref{IU2}{Registrar Reservacion}. 
			
		\end{Titemize}
	}\UCitem{Precondiciones}{\begin{Titemize}
			% Incluya Precondiciones lógicas, de negocio e incluso las que debe atender el usuario. 
			\Titem El cliente debe tener una cuenta	
			\Titem No debe haber otra mascota registrada con el mismo RUAC.
			\Titem La mascota debe contar con al menos las 3 vacunas solicitadas de acuerdo a \hyperlink{BR-002}{BR-002 Vacunas Obligatorias para la Mascota}
			% Muchas precondiciones provienen de reglas de negocios, otras estarán asociadas a manejo de errores
			% Otras están relacionadas con casos de uso que deben ejecutarse previamente, como registrar un producto.
		\end{Titemize}
	}\UCitem{Postcondiciones}{\begin{Titemize}
			\Titem Habrá una nueva mascota resgitrada para ese cliente en el sistema.
			\Titem La mascota aparecera en la lista de mascotas en la pantalla \IUref{IU5}{Regsitrar Reservacion}.
			\Titem El cliente podrá conservar los datos de la mascota registrada para reservaciones futuras.
			\Titem El cliente podrá realizar reservaciones.
			% Indique todas las postcondiciones
			% por ejemplo, Cambios en el sistema
			% Cambios en la BD una vez terminado el CU
			% Efectos colaterales
			% Condiciones de término.
		\end{Titemize}
	}\UCitem{Errores}{\begin{Titemize}
			% Escriba todos los errores que puedan ocurrir en el sistema, para cada error recuerde:
			% Punerle un identificador
			% Describir la condición o escenario que detona el error
			% Describa la forma en que debe reaccionar el sitema: si la reaccion corresponde a varios pasos use mejor una trayectoria alternativa.
			% Relacione el error con la trayectoria principal.
			\Titem {\bf \hypertarget{CU3.E1}{E1}}: Cuando el cliente no haya llenado algun campo obligatorio, el sistema muestra el mensaje \MSGref{MSG-001}{Campo obligatorio} y regresa al paso \ref{UC3.paso3}.
			\Titem {\bf \hypertarget{CU3.E2}{E2}}: Cuando el RUAC del cliente ya este registrado en el sistema, el sistema muestra el mensaje \MSGref{MSG-002}{RUAC ya existente} y regresa al paso \ref{UC3.paso3}.
.
			\Titem {\bf \hypertarget{CU3.E3}{E3}}: Cuando la mascota no cuenta con las 3 vacunas obligatorias, el sistema muestra el mensaje \MSGref{MSG-004}{Vacunas obligatorias} y regresa al paso \ref{UC3.paso3}
			\Titem {\bf \hypertarget{CU3.E4}{E4}}: Cuando el cliente ingrese un tipo de dato diferente al solicitado, el sistema muestra el mensaje \MSGref{MSG-007}{Dato no válido} y regresa al paso \ref{UC3.paso3}
			
		\end{Titemize}
	}\UCitem{Tipo}{
		% Especifique el tipo de caso de us, puede ser: "Caso de uso primario" o 
		% "Viene de \\hyperref{CUY}{CUY nombre del CU}" cuando se desprende desde otro caso de uso mediante un extends.
		Para ejecutar este caso de uso debe pasar primero por el \UCref{CUX}{Iniciar Sesion}. o \UCref{CU1}{Registrar Cliente} y estar en el caso de uso \UCref{CU2}{Regsitrar reservacion}
	}\UCitem{Observaciones}{
		% Indique las observaciones al caso de uso, las cuales pueden ser:
		Ninguna
		% - Dudas sobre el procedimiento o la especificación.
		% - Issues detectados
		% - Suposiciones realizadas.
		% - Cualquier otra especificacion que considere pertinente que no pudo colocarse en los demás atributos del Caso de uso
		% - Aclaraciones.
		% - Notas para el usuario o desarrollador.
		% - Pendientes (TODO's) en caso de no usar los comentarios.
	}
\end{UseCase}

%--------------------------------------
\begin{UCtrayectoria}
	% Cada paso debe inicair con un Verbo en infinitivo, siempre especificando el objetivo del paso mas la accion en concreto.
	% \UCpaso[\UCactor] se refiere al actor y \UCpaso se refiere al sistema.
	% A continuación viene ejemplos de pasos:
	% En el siguiente paso: "Ingresa al sistema" es el objetivo del paso y "escribiendo la URL de la aplicación" es la acción en concreto.
	\UCpaso[] \label{UC3.paso1} El cliente selecciona  el botón \IUbutton{Regsitrar nueva mascota} desde el formulario de la pantalla \IUref{IU2}{Registrar Reservacion}.
	\UCpaso [] \label{UC3.paso2} El sistema solicita los datos de la mascota mediante la pantalla  \IUref{IU3}{Registrar Mascota}.
	\UCpaso [] \label{UC3.paso3} El cliente proporciona los datos de su mascota.
	% En el siguiente paso está etiquetado para ser referenciado por un error o trayectoria alternativa:
	 \UCpaso[] \label{UC3.paso4} El cliente solicita el registro de la nueva mascota mediante el botón \IUbutton{Registrar} de la pantalla \IUref{IU2}{Registrar Mascota}.
	% En el siguiente paso se usa el comando \IUbutton 
\UCpaso[] {UC3.paso5}El sistema verifica que todos los campos obligatorios se hayan llenado \ErrorRef{CU3}{E1}{Campo obligatorio}.    \UCpaso[] {UC3.paso6} El sistema verifica que los datos introducidos tengan un formato válido en los campos correspondientes \ErrorRef{CU3}{E4}{Dato no válido}.
\UCpaso[] \label{UC3.paso7} El sistema verifica que no haya una mascota ya registrada con el mismo RUAC \ErrorRef{CU3}{E2}{RUAC ya existente}.
\UCpaso[] \label{UC3.paso8} El sistema verifica que la mascota cuente con las 3 vacunas obligatorias \ErrorRef{CU3}{E3}{Vacunas obligatoria}.
\UCpaso[] \label{UC3.paso9} El sistema registra la nueva mascota para ese cliente.
\UCpaso[] \label{UC3.paso10} El sistema muestra la pantalla \IUref{IU5}{Registar Reservacion} y la mascota esta disponible para ser seleccionada en la reservación.
	
\end{UCtrayectoria}


%--------------------------------------
% Las trayectorias alternativas se identifican con Letras: A, B, C, etc.

%--------------------------------------
% Puntos de extensión

% Comente la siguiente sección en caso de que no hayan puntos de extensión o relaciones de tipo extends.
\subsection{Puntos de extensión}
\UCExtenssionPoint{
	% Cuando se dá la extensión del Caso de uso:
El cliente quiere agregar un medicamento a su mascota.
}{
	% Durante la región (en que pasos se puede dar la extensión):
	Del paso \ref{UC3.paso3} al paso \ref{CUX.etiqueta}.
}{
	% Casos de uso a los que extiende:
	\UCref{CUZ}{Resgistrar Medicamento}.
}



%-------------------------------------- TERMINA descripción del caso de uso.