%!TEX root = ../proyecto.tex

% Plantilla para caso de uso sencillo con ejemplos de comandos e intrucciones.
%-------------------------------------- COMIENZA descripción del caso de uso.

%\begin{UseCase}[archivo de imágen]{UCX}{Nombre del Caso de uso}{
%--------------------------------------
\begin{UseCase}{CU3}{Registrar Mascota}{
		Cuando un cliente quiere realizar una reservación en el hotel, debe dar de alta en el sistema los datos de su mascota, lo puede realizar mientras realiza su reservación o posterior a la reservación, una vez registrados se agilizara su llegada al hotel el día de su check-in.
	}
	\UCitem{Versión}{\color{Gray}
		0.3	% Ponga un número de versión, 
	}\UCitem{Autor}{\color{Gray}
		Ruiz Evaristo Marco Antonio. % Analista responsable de especificar el CU
	}\UCitem{Supervisa}{\color{Gray}
		Hernández Jiménez Erick Yael. % Analista responsable de verificar que está correcto.
		% TODO: Dar de alta al actor Usuario
	}\UCitem{Actor}{
		\hyperlink{Cliente}{Cliente}% No olvide dar de alta el actor.
	}\UCitem{Propósito}{\begin{Titemize}%Indique los fines, objetivos, propósitos o valores agregados del Caso de uso.
		\Titem Facilitar al cliente la reservación de manera eficiente, ofreciendo un proceso intuitivo y simplificado para satisfacer sus necesidades.
		\Titem Asegurar que el cliente pueda registrar a su mascota durante o después de hacer la reservación, según lo prefiera, ofreciendo flexibilidad y conveniencia.
		\Titem Garantizar la seguridad y privacidad de los datos de las mascotas, además de proporcionar un sistema confiable que soporte las necesidades de los clientes al manejar eficientemente sus requerimientos.
		\Titem Almacenar los datos de las mascotas de forma segura para permitir reservaciones futuras, mejorando la experiencia del cliente al proporcionar un proceso de reservación más rápido y personalizado.
		\end{Titemize}
	}\UCitem{Entradas}{\begin{Titemize}
			% TODO: Dar de alta las entidades que se listan.
			\Titem \hyperlink{mascota.nombre}{Nombre} de la mascota. % El identificador no acepta acentos, espacios ni eñes.
			\Titem \hyperlink{mascota.tamano}{Tamaño} de la mascota. % Liste todos los datos de entrada
			\Titem \hyperlink{mascota.fechaNacimiento}{Fecha de nacimiento} de la mascota. 
			\Titem \hyperlink{mascota.CRUAC}{CRUAC} de la mascota.
			\Titem \hyperlink{mascota.especie}{especie} de la mascota.
			\Titem \hyperlink{mascota.raza}{raza} de la mascota.
			\Titem \hyperlink{mascota.foto}{fotografia} de la mascota.
			\Titem \hyperlink{mascota.tipoVacuna}{Tipo de vacuna}.
			\Titem \hyperlink{mascota.nombreVacuna}{Nombre de la vacuna}.
			\Titem \hyperlink{mascota.fechaAplicacion}{Ultima fecha de aplicación}.
			\Titem \hyperlink{mascota.dosis}{Numero de dosis aplicadas}.
			\Titem \hyperlink{mascota.alergias}{Alergias} de la mascota.
			\Titem \hyperlink{mascota.tratamientoMedico}{Tratamiento Medico}.
			\Titem \hyperlink{mascota.caracteristicasEspeciales}{Caracteristicas Especiales} fisicas o de comportamiento.
		\end{Titemize}
	}\UCitem{Origen}{\begin{Titemize}
			\Titem Se introducen desde el teclado. % Indique por que medio se introducen los datos, 
			\Titem Se seleccionan de un selector de fecha (calendario) con el mouse.
			\Titem Se seleccionan de un catalogo con el mouse.
			\Titem Se marca un checkbox con el mouse.
		
		\end{Titemize}
	}\UCitem{Salidas}{\begin{Titemize}
			% TODO: Dar de alta las entidades que se listan.
			\Titem Mensajes de error. % Indique por que medio se introducen los datos
			\Titem Datos de la mascota dentro de la lista de mascotas.
			\Titem Los datos de la mascota se envian al registro de la reservación. 
		\end{Titemize}
	}\UCitem{Destino}{\begin{Titemize}
			\Titem Se muestra en la pantalla \IUref{IU3}{Registrar Datos de la Mascota}.
			\Titem Se muestra en la pantalla \IUref{IU2}{Registrar Mascota}. 
			
		\end{Titemize}
	}\UCitem{Precondiciones}{\begin{Titemize}
			% Incluya Precondiciones lógicas, de negocio e incluso las que debe atender el usuario. 
			\Titem El cliente debe tener una cuenta	
			\Titem No debe haber otra mascota registrada con el mismo RUAC.
			\Titem La mascota debe contar con las vacunas solicitadas de acuerdo a \hyperlink{BR-002}{BR-002 Vacunas Obligatorias para la Mascota}
			% Muchas precondiciones provienen de reglas de negocios, otras estarán asociadas a manejo de errores
			% Otras están relacionadas con casos de uso que deben ejecutarse previamente, como registrar un producto.
		\end{Titemize}
	}\UCitem{Postcondiciones}{\begin{Titemize}
			\Titem Habrá una nueva mascota resgitrada para ese cliente en el sistema.
			\Titem La mascota aparecera en la lista de mascotas en la pantalla \IUref{IU2}{Registrar Mascota}.
			\Titem El cliente podrá conservar los datos de la mascota registrada para reservaciones futuras.
			\Titem El cliente podrá realizar reservaciones.
			% Indique todas las postcondiciones
			% por ejemplo, Cambios en el sistema
			% Cambios en la BD una vez terminado el CU
			% Efectos colaterales
			% Condiciones de término.
		\end{Titemize}
	}\UCitem{Errores}{\begin{Titemize}
			% Escriba todos los errores que puedan ocurrir en el sistema, para cada error recuerde:
			% Punerle un identificador
			% Describir la condición o escenario que detona el error
			% Describa la forma en que debe reaccionar el sitema: si la reaccion corresponde a varios pasos use mejor una trayectoria alternativa.
			% Relacione el error con la trayectoria principal.
			\Titem {\bf \hypertarget{CU3.E1}{E1}}: Cuando el cliente no haya llenado algun campo obligatorio, el sistema muestra el mensaje \MSGref{MSG-001}{Campo obligatorio} y regresa al paso \ref{UC3.formulario}.
			\Titem {\bf \hypertarget{CU3.E2}{E2}}: Cuando el RUAC del cliente ya este registrado en el sistema, el sistema muestra el mensaje \MSGref{MSG-002}{RUAC ya existente} y regresa al paso \ref{UC3.formulario}.
.
			\Titem {\bf \hypertarget{CU3.E3}{E3}}: Cuando la mascota no cuenta con las vacunas obligatorias, el sistema muestra el mensaje \MSGref{MSG-004}{Vacunas obligatorias} y regresa al paso \ref{UC3.formulario}
			\Titem {\bf \hypertarget{CU3.E4}{E4}}: Cuando el cliente ingrese un tipo de dato diferente al solicitado, el sistema muestra el mensaje \MSGref{MSG-007}{Dato no válido} y regresa al paso \ref{UC3.formulario}
			\Titem {\bf \hypertarget{CU3.E5}{E5}}: Cuando el cliente ingrese un tamaño diferente al antes seleccionado en la reservación, el sistema muestra el mensaje \MSGref{MSG-009}{Tamaño no valido (si deseas cambiar el tamaño de la mascota esto prodría modificar el precio de la reservación)} y continua con el paso \ref{UC3.datos de la mascota}
			\Titem {\bf \hypertarget{CU3.E6}{E6}}: Cuando el cliente ingrese una raza diferente a la antes seleccionada en la reservación, el sistema muestra el mensaje \MSGref{MSG-010}{Raza no valida (si deseas cambiar el tamaño de la mascota esto prodría modificar el precio de la reservación)} y continua con el paso \ref{UC3.datos de la mascota}
			\Titem {\bf \hypertarget{CU3.E7}{E7}}: Cuando el cliente ingrese una especie diferente a la antes seleccionada en la reservación, el sistema muestra el mensaje \MSGref{MSG-011}{Especie no valida (si deseas cambiar el tamaño de la mascota esto prodría modificar el precio de la reservación)} y continua con el paso \ref{UC3.datos de la mascota}
			\Titem {\bf \hypertarget{CU3.E8}{E8}}: Cuando el el tiempo de aplicacion de la vacuna exceda la fecha de check out, el sistema muestra el mensaje \MSGref{MSG-012}{La aplicacion de la vacuna expira antes o durante su estancia}  y regresa al paso \ref{UC3.formulario}
			\Titem {\bf \hypertarget{CU3.E9}{E9}}: Cuando los datos que se quieren registrar coinciden un 80\% en los datos de un registro que ya existente, el sistema muestra el mensaje \MSGref{MSG-020}{Registro ya existente}  y continua con el paso \ref{UC3.registrar} o termina el caso de uso.
			
		\end{Titemize}
	}\UCitem{Tipo}{
		% Especifique el tipo de caso de us, puede ser: "Caso de uso primario" o 
		% "Viene de \\hyperref{CUY}{CUY nombre del CU}" cuando se desprende desde otro caso de uso mediante un extends.
		Para ejecutar este caso de uso debe pasar primero por el \UCref{CUX}{Iniciar Sesion} o \UCref{CU1}{Registrar Cliente} y  el caso de uso \UCref{CU2}{Regsitrar reservacion}
	}\UCitem{Observaciones}{
		% Indique las observaciones al caso de uso, las cuales pueden ser:
		Ninguna
		% - Dudas sobre el procedimiento o la especificación.
		% - Issues detectados
		% - Suposiciones realizadas.
		% - Cualquier otra especificacion que considere pertinente que no pudo colocarse en los demás atributos del Caso de uso
		% - Aclaraciones.
		% - Notas para el usuario o desarrollador.
		% - Pendientes (TODO's) en caso de no usar los comentarios.
	}
\end{UseCase}

%--------------------------------------
\begin{UCtrayectoria}
	% Cada paso debe inicair con un Verbo en infinitivo, siempre especificando el objetivo del paso mas la accion en concreto.
	% \UCpaso[\UCactor] se refiere al actor y \UCpaso se refiere al sistema.
	% A continuación viene ejemplos de pasos:
	% En el siguiente paso: "Ingresa al sistema" es el objetivo del paso y "escribiendo la URL de la aplicación" es la acción en concreto.
	\UCpaso[\UCactor] \label{UC3.detonante} El cliente selecciona  el botón \IUbutton{Registrar nueva mascota} desde el formulario de la pantalla \IUref{IU2}{Registrar Mascota}.
	\UCpaso [\UCsist] \label{UC3.formulario} El sistema solicita los datos de la mascota mediante la pantalla  \IUref{IU3}{Registrar datos de la Mascota}.
	\UCpaso [\UCactor] \label{UC3.datos de la mascota} El cliente proporciona los datos de su mascota.
	\UCpaso [\UCactor] \label{UC3.CRUAC} El cliente indica si su mascota cuenta con CRUAC \Trayref{CU3}{C}.
	\UCpaso [\UCactor] \label{UC3.vacunas obligatorias} El cliente selecciona las vacunas obligatorias con las que cuenta su mascota \Trayref{CU3}{B}.
	% En el siguiente paso está etiquetado para ser referenciado por un error o trayectoria alternativa:
	 \UCpaso[\UCactor] \label{UC3.solictar registrar} El cliente solicita el registro de la nueva mascota mediante el botón \IUbutton{Registrar} de la pantalla \IUref{IU2}{Registrar Mascota}.
	% En el siguiente paso se usa el comando \IUbutton 
\UCpaso[\UCsist]\label {UC3.verificacion campo obligatorio}El sistema verifica que todos los campos obligatorios se hayan llenado \ErrorRef{CU3}{E1}{Campo obligatorio}.   
\UCpaso[\UCsist] \label{UC3.verificacion formato invalido} El sistema verifica que los datos introducidos tengan un formato válido en los campos correspondientes \ErrorRef{CU3}{E4}{Dato no válido}.
\UCpaso[\UCsist] \label{UC3.verificacion RUAC} El sistema verifica que no haya una mascota ya registrada con el mismo RUAC \ErrorRef{CU3}{E2}{RUAC ya existente}.
\UCpaso[\UCsist] \label{UC3.verificacion vacunas obligatorias} El sistema verifica que la mascota cuente con las vacunas obligatorias \ErrorRef{CU3}{E3}{Vacunas obligatoria}.
\UCpaso[\UCsist] \label{UC3.verificacion duracion vacuna} El sistema verifica que el tiempo de aplicacion de la vacuna no supere la fecha de check out \ErrorRef{CU3}{E8}{La aplicacion de la vacuna expira antes o durante su estancia}.
\UCpaso[\UCsist] \label{UC3.verificacion tamaño} El sistema verifica que el campo de tamaño no haya sido modificado \ErrorRef{CU3}{E5}{Tamaño no valido(si deseas cambiar el tamaño de la mascota esto prodría modificar el precio de la reservación)}.
\UCpaso[\UCsist] \label{UC3.verificacion raza} El sistema verifica que el campo de la raza no haya sido modificado \ErrorRef{CU3}{E6}{Raza no valida (si deseas cambiar el tamaño de la mascota esto prodría modificar el precio de la reservación)}.
\UCpaso[\UCsist] \label{UC3.verificacion especie} El sistema verifica que el campo de la especie no haya sido modificado \ErrorRef{CU3}{E7}{Especie no valida (si deseas cambiar el tamaño de la mascota esto prodría modificar el precio de la reservación)}.
\UCpaso[\UCsist] \label{UC3.verificacion duplicidad de datos} El sistema verifica que los datos que se quieren registrar no coincidan en un 80\% con los datos de un registro que ya existente el sistema \ErrorRef{CU3}{E9}{Registro ya existente}.
\UCpaso[\UCsist] \label{UC3.registrar} El sistema registra la nueva mascota para ese cliente.
\UCpaso[\UCsist] \label{UC3.seleccionar mascota} El sistema muestra la pantalla \IUref{IU2}{Registar Mascota} y la mascota esta disponible para ser seleccionada en la reservación.
	
\end{UCtrayectoria}


%--------------------------------------
% Las trayectorias alternativas se identifican con Letras: A, B, C, etc.
\begin{UCtrayectoriaA}{CU3}{A}{Cuando la edad de la mascota sea mayor a un año}
	\UCpaso[\UCsist] El sistema bloqueara las vacunas que ya no es necesario que tenga la mascota.
	\UCpaso[\UCactor] El cliente proporciona los datos necesarios para las vacunas.\Trayref{CU3}{B}
	% Se puede desprender otra trayectria alternativa si es necesario.
	% Finalice la trayectoria indicando si la ejecución se integra a la trayectoria anterior o si termina la ejecución del CU.
	% Verifique que la redacción de la trayectoria deje en claro si el objetivo del CU se alcanzó o no.
	\UCpaso[] El Caso de uso continúa en el paso \ref{UC3.formulario}.
\end{UCtrayectoriaA}

\begin{UCtrayectoriaA}{CU3}{B}{Cuando el cliente seleccione las vacunas obligatorias que tiene su mascota}
	\UCpaso [\UCsist]El sistema desplegara 3 campos más para complentar la información de las vacunas aplicadas los cuales son Nombre de la vacuna, Ultima fecha de aplicación, Numero de dosis aplicadas.
	\UCpaso [\UCactor]El cliente proporciona los datos de las vacunas de su mascota.
	% Se puede desprender otra trayectria alternativa si es necesario.
	% Finalice la trayectoria indicando si la ejecución se integra a la trayectoria anterior o si termina la ejecución del CU.
	% Verifique que la redacción de la trayectoria deje en claro si el objetivo del CU se alcanzó o no.
	\UCpaso[] El Caso de uso continúa en el paso \ref{UC3.formulario}.
\end{UCtrayectoriaA}

\begin{UCtrayectoriaA}{CU3}{C}{Cuando el cliente indique que su mascota cuanta con su CRUAC}
	\UCpaso [\UCsist]El sistema desplegara un campo mas para poder capturar el CRUAC.
	\UCpaso [\UCactor]El cliente proporciona el CRUAC de la mascota.
	% Se puede desprender otra trayectria alternativa si es necesario.
	% Finalice la trayectoria indicando si la ejecución se integra a la trayectoria anterior o si termina la ejecución del CU.
	% Verifique que la redacción de la trayectoria deje en claro si el objetivo del CU se alcanzó o no.
	\UCpaso[] El Caso de uso continúa en el paso \ref{UC3.formulario}.
\end{UCtrayectoriaA}


\begin{UCtrayectoriaA}{CU3}{D}{Cuando el cliente deseé registrar a su mascota posterior a la realización de la reservación}
	\UCpaso[\UCactor] El cliente selecciona  el botón \IUbutton{Regsitrar mascota} desde la consulta de su reservación en la pantalla \IUref{IU5}{Home Cliente}.
	\UCpaso [\UCsist] El sistema muestra la pantalla  \IUref{IU2}{Registrar Mascota} y da la opción de seleccionar una mascota ya registrada o registrar una nueva mascota con el botón \IUbutton{Registrar nueva mascota}
	% Se puede desprender otra trayectria alternativa si es necesario.
	% Finalice la trayectoria indicando si la ejecución se integra a la trayectoria anterior o si termina la ejecución del CU.
	% Verifique que la redacción de la trayectoria deje en claro si el objetivo del CU se alcanzó o no.
	\UCpaso[] El Caso de Uso continúa en el paso \ref{UC3.detonante}.
\end{UCtrayectoriaA}


%--------------------------------------
% Puntos de extensión

% Comente la siguiente sección en caso de que no hayan puntos de extensión o relaciones de tipo extends.
\subsection{Puntos de extensión}
\UCExtenssionPoint{
	% Cuando se dá la extensión del Caso de uso:
El cliente quiere agregar una alergía a su mascota.
}{
	% Durante la región (en que pasos se puede dar la extensión):
	Del paso \ref{UC3.datos de la mascota} al paso \ref{CUX.etiqueta}.
}{
	% Casos de uso a los que extiende:
	\UCref{CUZ}{Resgistrar Alergia}.
}

\subsection{Puntos de extensión}
\UCExtenssionPoint{
	% Cuando se dá la extensión del Caso de uso:
El cliente quiere agregar un tratamiento medico a su mascota.
}{
	% Durante la región (en que pasos se puede dar la extensión):
	Del paso \ref{UC3.datos de la mascota} al paso \ref{CUX.etiqueta}.
}{
	% Casos de uso a los que extiende:
	\UCref{CUZ}{Resgistrar Tratamiento Medico}.
}




%-------------------------------------- TERMINA descripción del caso de uso.